\usepackage[left=2cm, right=2cm, top=2cm, bottom=2cm, bindingoffset=0cm]{geometry}
\usepackage{cmap} % поддержка Unicode при компиляции в PDF
\usepackage{amsmath, amsthm, amssymb, mathtext} % математические пакеты
\usepackage{physics} % содержит дифференциальные операторы (\dd, \dv, \pdv и т. д.) и не только
\usepackage[T1, T2A]{fontenc}
\usepackage{dsfont}
\usepackage[utf8]{inputenc}
\usepackage[english, russian]{babel}
\usepackage{fancybox, fancyhdr}
\usepackage{ulem, color, graphicx, tcolorbox}
\usepackage{float, wrapfig, subcaption}
\usepackage{setspace} % setstretch
\usepackage{enumerate} % дополнительный функционал для списков
\usepackage{multicol} % разделение на несколько колонок
\usepackage[explicit, compact]{titlesec} % изменение стиля заголовков (главы, разделы и т. д.)
\usepackage{changepage} % margin adjustment and detection of odd/even pages
\usepackage{needspace} % если заголовок находится в самом низу страницы, он переносится на новую
\usepackage{ifthen} % условия
\usepackage{tabularx} % таблицы
\usepackage[bottom]{footmisc} % "прикрепление" сносок к низу страницы
\usepackage{mdframed} % frames

% настройка изображений
\DeclareGraphicsExtensions{.pdf, .png, .jpg}
\graphicspath{{./img/}}

\tcbuselibrary{theorems} % подключение помпезных стилей для формул, теорем, лемм

% \setlength\parindent{1.25cm} % отступ 1.25
% \setstretch{1.5} % интервал 1.5