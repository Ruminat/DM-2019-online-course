\providecommand{\C}{\mathbb C} % множество комплексных чисел    (C)
\providecommand{\R}{\mathbb R} % множество действительных чисел (R)
\providecommand{\Q}{\mathbb Q} % множество рациональных чисел   (Q)
\providecommand{\Z}{\mathbb Z} % множество целых чисел          (Z)
\providecommand{\N}{\mathbb N} % множество натуральных чисел    (N)
\providecommand{\Uni}{\mathbb U} % универсальное множество        (U)

\newcommand{\Set}[2]{\qty{#1 \mid #2}}           % множество {x \in \R | x^2 < 0} {before} | {after}
\newcommand{\Vector}[1]{\left\langle#1\right\rangle}
\newcommand{\sq}[2][1]{\qty{#2}_{k=#1}^{\infty}} % последовательность
% \newcommand{\SQ}[2][1]{\qty{#2}_{k=#1}^\infty}
\newcommand{\fn}[3][f]{#1 \colon #2 \to #3}      % отображение f: R -> R [f]{R}{R}

% символ \ (вычитание множества и делимость в теории чисел (a \ b --- a делит b))
\newcommand{\setm}{\setminus}
\newcommand{\NOT}[1]{\overline{#1}} % отрицание (черта над формулой)