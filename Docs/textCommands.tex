\renewcommand*{\thefootnote}{[\arabic{footnote}]} % изменение вида сноски с 1 на [1]
\providecommand{\code}[1]{{\smaller \texttt{#1}}} % шрифт для кода


% --- Таблицы (tabularx)
\newcolumntype{L}{>{\raggedright\arraybackslash}X} % колонка, выравниваемая по левой стороне
\newcolumntype{C}{>{\centering\arraybackslash}X}   % колонка, выравниваемая по центру
\newcolumntype{R}{>{\raggedleft\arraybackslash}X}  % колонка, выравниваемая по правой стороне


% --- Изображения
% шаблон для изображений
\providecommand{\imgs}[1]{%
	\begin{figure}[H]%
		\centering%
		#1%
	\end{figure}%
}
% изображение без названия и метки размера a[=1]
% (a --- коэффициент перед \textwidth => 0 < a <= 1) [a]{path/to/image.png}
\providecommand{\img}[2][1]{%
	\imgs{\includegraphics[width=#1\textwidth]{#2}}%
}
% изображение с названием, но без метки, размера a[=1] [a]{path/to/image.png}{The Most beautiful picture}
\providecommand{\imgt}[3][1]{%
	\imgs{%
		\includegraphics[width=#1\textwidth]{#3}%
		\caption{#2}%
	}%
}
% изображение с названием и с меткой размера a[=1]
% [a]{path/to/image.png}{The Most beautiful picture}{pic:beautiful}
\providecommand{\imgl}[4][1]{
	\imgs{
		\includegraphics[width=#1\textwidth]{#3}
		\caption{#2}
		\label{#4}
	}
}
% (под)изображение (в окружении figure) размера a[=1]
% с названием, но без метки [a]{path/to/image.png}{The Most beautiful picture}
\providecommand{\subimg}[3][0.45]{
	\begin{subfigure}[H]{#1\textwidth}
		\includegraphics[width=\textwidth]{#3}
		\caption{#2}
	\end{subfigure}
}
% (под)изображение (в окружении figure) размера a[=1]
% с названием и с меткой [a]{path/to/image.png}{The Most beautiful picture}{pic:beautiful}
\providecommand{\subimgl}[4][0.45]{
	\begin{subfigure}[H]{#1\textwidth}
		\includegraphics[width=\textwidth]{#3}
		\caption{#2}
		\label{#4}
	\end{subfigure}
}