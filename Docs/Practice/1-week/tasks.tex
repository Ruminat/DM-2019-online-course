%!TeX root = ./Практика.tex
\task{%
  При условии, что $ A \subseteq B $, докажите:
  \begin{enumerate}[a)]%
    \item $ A \cup B = B $,
    \item $ A \cap B = A $,
    \item $ A \divisionsymbol B = B \setminus A $.
  \end{enumerate}
}

\solution{%
  Опустим тривиальные случаи, когда одно (или оба) из множеств:~$A$~или~$B$, "--- является пустым.
  \begin{enumerate}[a)]%
    \item Так как любой элемент, находящийся в~$A$, находится и в~$B$, то при объединении~$A$~с~$B$, что равносильно~$B \cup A$ (по свойству коммутативности), мы сначала возьмём все элементы из~$B$, после чего добавим к ним ещё какие-то элементы из~$B$ (то есть, элементы из~$A$), что, очевидно, даст нам в результате множество~$B$. \qed
    \item Рассуждая подобным образом: мы должны взять элементы, которые находятся в обоих множествах, то есть все элементы из~$A$. \qed
    \item По определению:
      \[
        A \divisionsymbol B = \underbrace{(A \cup B)}_{B} \setminus \underbrace{(A \cap B)}_{A} = B \setminus A. \qed
      \]
  \end{enumerate}

  Данное доказательство может быть проиллюстрировано с помощью диаграммы Эйлера"--~Венна. И, хотя строгим математическим доказательством эту иллюстрацию назвать нельзя, доказанные равенства на её фоне могут показаться довольно очевидными.
  % \imgt[0.3]{Иллюстрация доказательства}{Euler.png}
  \imgs{
    \caption{Иллюстрация доказательства диаграммами Эйлера"--~Венна}
    \subimg[0.3]{$ A \cup B = B $}{B.png}
    \subimg[0.3]{$ A \cap B = A $}{A.png}
    \subimg[0.3]{$ A \divisionsymbol B = B \setminus A $}{B-A.png}
  }
}


\task{%
  Доказать равенство:
  \begin{equation}\label{somebodyOnceToldMe}
    \NOT{A \cap B} = \n A \cup \n B.
  \end{equation}
}

\solution{%
  Множества~$A$~и~$B$ равны тогда и только тогда, когда $A$ "--- подмножество~$B$, и $B$ "--- подмножество~$A$, то есть:
  \begin{equation}
    A = B \iff A \subseteq B \text{ и } B \subseteq A.
  \end{equation}
  Иными словами, для доказательства \eqref{somebodyOnceToldMe} нам достаточно показать, что
  \[
    \underbrace{\qty(\NOT{A \cap B}) \subseteq \qty(\n A \cup \n B)}_{1.} \text{ и }
    \underbrace{\qty(\n A \cup \n B) \subseteq \qty(\NOT{A \cap B})}_{2.}.
  \]
  \begin{enumerate}[1.]%
    \item Пусть $ x \in \NOT{A \cap B} $,
      тогда $ x \in \Uni \setminus (A \cap B) $, а значит $ x \in \Uni $ и $ x \not\in A \cap B $,
      следовательно
      \[
        x \in \Uni \text{ и } x \not\in A \text{ или } x \in \Uni \text{ и } x \not\in B.
      \]
      \begin{itemize}%
        \item Если $ x \in \Uni \text{ и } x \not\in A $, то $ x \in \n A $ и, следовательно, $ x \in \n A \cup \n B $.
        \item Аналогично доказывается случай с $ x \in \Uni \text{ и } x \not\in B $.
      \end{itemize}
    \item Пусть $ x \in \n A \cup \n B $, тогда, по определению операции объединения, $ x \in \n A $ или $ x \in \n B $.
      \begin{itemize}%
        \item Если $ x \in \n A $, тогда $ x \in \Uni \setminus A $, а значит $ x \in \Uni $ и $ x \not\in A $. Если элемент не принадлежит какому-либо множеству~$C$, то он не будет принадлежать и пересечению данного множества~$C$ с любым другим множеством, то есть $ x \not\in A \Rightarrow x \not\in A \cap B $, а это значит, что если $ x \in \Uni \setminus (A \cap B) $, то $ x \in \NOT{A \cap B} $.
        \item Аналогично доказывается случай с $ x \in \n B $. \qed
      \end{itemize}
  \end{enumerate}
}


\task{%
  Известно, что натуральное число $n$ принадлежит множеству~$D$ тогда и только тогда, когда количество его делителей чётно. То есть, например, числа $2$,~$3$~и~$5$ принадлежат~$D$, а $1$~и~$4$ "--- нет. Дано множество~$P$, которое является множеством простых чисел, и множество~$S$:
  \begin{equation}
    S = \qty{n \in \N \mid n \leq 2^{1729}}.
  \end{equation}
  Определите, в каком из двух множеств:~$A$~и~$Z$, "--- больше элементов (или, если в обоих множествах количество элементов одинаково, укажите это):
  \begin{enumerate}[a)]%
    \item $ A = S \cap P, \, Z = S \cap (D \cap P) $;
    \item $ A = S \cap (P \divisionsymbol D), \, Z = S \cap (D \cup P) $;
    \item $ A = S \cap (P \divisionsymbol (D \setminus \N)), \, Z = S \cap (D \cup P \cap D) $.
  \end{enumerate}
}

\solution{%
  Заметим, что~$ P \subseteq D $, так как любое простое число~$p$ имеет 2 делителя:~$1$~и~$p$.
  \begin{enumerate}[a)]%
    \item $ P \subseteq D \Rightarrow D \cap P = P \Rightarrow A = B \Rightarrow \text{количество элементов одинаково.} $
    \item Довольно очевидно, что $ (P \divisionsymbol D) \subseteq (D \cup P) $ "--- это значит, что в~$B$ элементов не меньше, чем в~$A$. Рассмотрим число~$2$, которое, конечно, не принадлежит~$P \divisionsymbol D$, а значит, не принадлежит и~$A$, но, с другой стороны, оно, естественно, принадлежит~$B$, что говорит нам о том, что в~$B$ элементов больше, чем в~$A$.
    \item Понятно, что $ D \setminus \N = \varnothing $, нетрудно догадаться, что~$ P \divisionsymbol \varnothing = P $, то есть~$ A = S \cap P $. Заметим, что~$ D \cup P \cap D = D $, то есть~$ B = S \cap D $. Как уже было сказано ранее,~$ P \subseteq D $, причём число~$6$ принадлежит лишь одному из этих двух множеств "--- множеству~$D$, другими словами, в~$B$ элементов больше, чем в~$A$.
  \end{enumerate}
  \answer{a) Количество элементов одинаково. \ b, c) В~$B$ элементов больше.}
}


\task{%
  Найдите наименьшее положительное значение параметра~$\sigma$, при котором количество подмножеств множества~$S \cap L$ будет равно~$4^t$, где~$t$ "--- натуральное число:
  \begin{equation}
    \begin{array}{c}%
      S = \qty{n \in \N \mid n = 1 + 2 + 3 + \ldots + k \text{, где $k$ "--- некоторое натуральное число}}, \\ 
      L = \qty{n \in \N \mid n \leq \sigma}.
    \end{array}
  \end{equation}
}

\solution{%
  Как известно, у множества из~$n$ элементов имеется~$ 2^n $ подмножеств, то есть, нам нужно подобрать такое значение~$\sigma$, что у множества~$ S \cap L $ будет $ 2t $~элементов ($ 2^{2t} = 4^t $).

  Рассмотрим для начала, как выглядят элементы множеств~$S$~и~$L$. Для~$L$ всё довольно просто:
  \begin{equation}
    L = \qty{1, \, 2, \, 3, \, \ldots, \, \sigma}.
  \end{equation}
  Что же касается~$S$, то тут нам нужно вспомнить формулу суммы арифметической прогрессии:
  \begin{equation}\label{sum}%
    1 + 2 + 3 + \ldots + k = \frac{k (k + 1)}{2}.
  \end{equation}
  То есть, множество~$S$ имеет вид:
  \begin{equation}
    S = \qty{1, \, 3, \, 6, \, 10, \, \ldots, \, \frac{k (k + 1)}{2}, \, \ldots}.
  \end{equation}
  
  Понятно, что любое число из~$S$ находится и в~$L$ (если взять достаточно большое~$\sigma$). То есть, нам нужно ограничить множество~$S$ таким образом, чтобы в нём осталось лишь~$2t$ элементов. Судя по формуле~\eqref{sum}, минимальным положительным значением~$\sigma$ будет $\frac{1}{2} (2t (2t + 1))$.
  \answer{$\sigma = \frac{1}{2} (2t (2t + 1))$.}
}


\task{%
  Даны множества~$A$~и~$B$:
  \begin{equation}
    A = \Z \divisionsymbol \qty{1, \, 1 + \frac{1}{2}, \, 1 + \frac{3}{4}, \, 1 + \frac{7}{8}, \, \ldots}, \
    B = \qty{0, \, \frac{1}{100}, \, \frac{2}{100}, \, \frac{3}{100}, \, \ldots}.
  \end{equation}
  Найдите~$ A \setminus B $.
}

\solution{%
  Рассмотрим сначала последовательность чисел:
  \[
    \qty(1, \, 1 + \frac{1}{2}, \, 1 + \frac{3}{4}, \, 1 + \frac{7}{8}, \, \ldots).
  \]
  Стремится она, очевидно, к числу $2$, однако само это число не содержит, следовательно, множество~$A$ может быть представлено следующим образом:
  \begin{equation}
    A = (\Z \setminus \qty{1}) \cup \qty{1 + \frac{1}{2}, \, 1 + \frac{3}{4}, \, 1 + \frac{7}{8}, \, \ldots}.
  \end{equation}
  Множество~$B$ же содержит числа вида:
  \begin{equation}
    \frac{n}{100}, \, \text{ где $n$ "--- целое неотрицательное число}.
  \end{equation}
  Исключив из~$A$ элементы множества~$B$, получим ответ.
  \answer{$ A \setminus B = \qty{n \in \Z \mid n < 0} \cup \qty{1 + \frac{1}{2}, \, 1 + \frac{3}{4}, \, 1 + \frac{7}{8}, \, \ldots} $.}
}


\task{%
  Докажите тождество:
  \begin{equation}
    \NOT{\n A \cap \n B \cap \n C} = (A \cup B \cap A) \cup B \cup \qty(\qty(C \setminus \qty(A \cap B \cap \NOT{\n A \cap \n B \cap \n C})) \cup C).
  \end{equation}
}

\solution{%
  Чтобы доказать тождество, достаточно привести одну часть к другой, либо же упростить оба выражения и убедиться, что они идентичны. Воспользуемся вторым методом. Упростим сначала первое выражение, использовав закон двойственности де~Моргана, а так же свойство~$\bar{\bar{A}} = A$:
  \begin{equation}\label{t1-a}%
    \NOT{\n A \cap \n B \cap \n C} = A \cup \NOT{\n B \cap \n C} = A \cup B \cup C.
  \end{equation}
  Осталось лишь привести второе выражение к виду \eqref{t1-a}. Рассмотрим второе выражение, выделив части (a)~и~(b):
  \[
    \underbrace{(A \cup B \cap A)}_{(a)} \cup B \cup \underbrace{\qty(\qty(C \setminus \qty(A \cap B \cap \NOT{\n A \cap \n B \cap \n C})) \cup C)}_{(b)}.
  \]
  \begin{itemize}%
    \item[$(a)$] Заметим, что в выражении~$ A \cup B \cap A $ происходит объединение множества~$A$ с вложенным в него множеством~$ B \cap A $, что, разумеется, в результате даст множество~$A$.
    \item[$(b)$] Здесь же неважно, что вычитается из множества~$C$, так как далее происходит объединение с этим самым~$C$, то есть, в итоге мы получаем~$C$.
  \end{itemize}
  Таким образом, второе выражение упрощается до следующего:
  \begin{equation}
    A \cup B \cup C,
  \end{equation}
  которое, как можно заметить, равно первому. \qed
}


\task{%
  Множество~$Q$ определено следующим образом:
  \begin{equation}\label{t2-e}
    Q = \qty{x \in \R \mid \sqrt{1 - x^2} \geq \sqrt[4]{1 - x}}.
  \end{equation}
  Запишите множество~$Q$ в виде объединения точек/отрезков/интервалов.
}

\solution{%
  По выражению \eqref{t2-e} можно сказать, что множество~$Q$ представляет собой множество решений неравенства:
  \[
    \sqrt{1 - x^2} \geq \sqrt[4]{1 - x}
  \]
  относительно вещественной переменной~$x$. Чтобы решить это неравенство, необходимо для начала найти множество допустимых значений~$x$ (ОДЗ). Как известно, функция~$\sqrt[2n]{x}$, где~$n$ "--- некоторое натуральное число, определена лишь при неотрицательном~$x$. Значит, ОДЗ определяется как~$[-1; \ 1] \cap (-\infty; \ 1] = [-1; \ 1]$. Теперь возведём обе части неравенства в четвёртую степень, после чего упростим его:
  \[
    (1 - x^2)^2 \geq 1 - x \Leftrightarrow x (x^3 - 2x + 1) \geq 0.
  \]
  
  Найдём корни уравнения $x (x^3 - 2x + 1) = 0$. Очевидно, что один из корней "--- 0. Как известно, многочлен с целыми коэффициентами имеет как минимум один целый корень, который делит свободный член~(в данном случае, 1). Перебираем возможные делители (то есть, $1$ и $-1$), получаем корень 1. То есть, получаем следующее уравнение:
  \[
    x (x - 1) (x^2 + x - 1) = 0.
  \]
  Решаем квадратное уравнение и получаем ещё 2 корня:
  \[
    x_1 = \underbrace{\frac{-1 - \sqrt{5}}{2}}_{\approx -1.618}, \
    x_2 = \underbrace{\frac{-1 + \sqrt{5}}{2}}_{\approx 0.618}.
  \]

  Дальнейшее решение неравенства тривиально (можно, например, применить метод интервалов, не забыв при этом про ОДЗ), поэтому просто выпишем ответ.
  \answer{$ Q = \qty[0; \  \frac{1}{2} \qty(-1 + \sqrt{5})] \cup \{1\} $.}
}


\task{%
  Верны ли следующие утверждения?
  \begin{enumerate}[a)]%
    \item $ \qty{a, \, i, \, u, \, e, \, o} = \qty{\qty{a, \, i, \, u, \, e, \, o}} $,
    \item $ \qty{\qty{a}, \, \qty{i}, \, \qty{u}, \, \qty{e}, \, \qty{o}} = \qty{a, \, i, \, u, \, e, \, o} $,
    \item $ \qty{a, \, \qty{i, \, \qty{u, \, \qty{e, \, \qty{o}}}}} = \qty{\qty{\qty{\qty{\qty{o}, \, e}, \, u}, \, i}, \, a} $.
  \end{enumerate}
}

\solution{%
  \begin{enumerate}[a)]%
    \item Разумеется, нет, так как первое множество содержит 5 элементов, а второе "--- 1 (оно содержит одно множество из пяти элементов).
    \item Опять же, нет, так как $ \qty{a} $~и~$ a $ суть разные вещи. Первое есть ничто иное, как \textit{множество}, содержащее некий элемент~$a$, а второе "--- сам \textit{элемент}~$a$.
    \item А вот здесь уже другой случай "--- второе множество просто содержит элементы первого в другом порядке, но множество "--- не последовательность, то есть порядок элементов не имеет значения. То есть, множества равны.
    \begin{equation}
      \begin{array}{ccl}%
         &   & \qty{\qty{\qty{\qty{\qty{o}, \, e}, \, u}, \, i}, \, a}  = \\ 
         & = & \qty{a, \, \qty{\qty{\qty{\qty{o}, \, e}, \, u}, \, i}}  = \\ 
         & = & \qty{a, \, \qty{i, \, \qty{\qty{\qty{o}, \, e}, \, u}}}  = \\ 
         & = & \qty{a, \, \qty{i, \, \qty{u, \, \qty{\qty{o}, \, e}}}}  = \\ 
         & = & \qty{a, \, \qty{i, \, \qty{u, \, \qty{e, \, \qty{o}}}}}.   \\ 
      \end{array}
    \end{equation}
  \end{enumerate}
  \answer{a, b) Нет. \ c) Да.}
}


\task{%
  Даны множества~$ A, \, B, \, C, \, \ldots, \, Z $:
  \begin{equation}
    A = \qty{a}, \, B = \qty{a, \, b}, \, C = \qty{a, \, b, \, c}, \, \ldots, \, Z = \qty{a, \, b, \, c, \, \ldots, \, z}.
  \end{equation}
  Дано так же множество~$ \Omega = \qty{A, \, B, \, C, \, \ldots, \, Z} $. Для каждого из множеств: $\Gamma_1$, $\Gamma_2$, $\Gamma_3$, "--- сосчитайте количество подмножеств:
  \begin{itemize}%
    \item $ \Gamma_1 = A \cup B \cup C \cup \ldots \cup Z $,
    \item $ \Gamma_2 = A \divisionsymbol B \divisionsymbol C \divisionsymbol \ldots \divisionsymbol Z $,
    \item $ \Gamma_3 = \Gamma_2 \cup (\Omega \setminus \Gamma_1) $.
  \end{itemize}
}

\solution{%
  \begin{itemize}%
    \item Понятно, что~$ A \subset B \subset C \subset \ldots \subset Z $, поэтому объединение всех этих множеств даст нам~$Z$. То есть,~$ \Gamma_1 = Z $, количество подмножеств равно~$ 2^{26} $ (в английском алфавите 26~букв).
    \item Обобщим операцию симметричной разности ($\divisionsymbol$) на~$n$ множеств. Выпишем несколько первых результатов:
      \begin{equation}
        \begin{array}{rcl}%
          A & = & \qty{a}, \\ 
          A \divisionsymbol B & = & \qty{b}, \\ 
          A \divisionsymbol B \divisionsymbol C & = & \qty{a, \, c}, \\ 
          A \divisionsymbol B \divisionsymbol C \divisionsymbol D & = & \qty{b, \, d}, \\ 
          A \divisionsymbol B \divisionsymbol C \divisionsymbol D \divisionsymbol E & = & \qty{a, \, c, \, e}. \\ 
        \end{array}
      \end{equation}
      Можно заметить, что каждый элемент появляется «через раз», то есть если мы применяем операцию симметричной разности к~$n$~множествам, то определить, входит какой-либо элемент~$\alpha$ в результат или нет можно следующим образом: если в сумме во всех $n$~множествах он встретился нечётное количество раз, то мы включаем его в результат, иначе "--- нет.

      Таким образом, следуя вышеуказанной логике и учитывая, что элемент~$a$ встречается нам 26 раз,~$b$ "--- 25,~$c$ "--- 24,~\ldots,~$z$ "--- 1, получаем:
      \begin{equation}
        \Gamma_2 = \qty{b, \, d, \, f, \, h, \, j, \, l, \, n, \, p, \, r, \, t, \, v, \, x, \, z}.
      \end{equation}
      Количество подмножеств "--- $ 2^{13} $.
    \item Ясно, что $ \Omega \setminus \Gamma_1 = \Omega $, так как $\Omega$~состоит из множеств~$A, \, \ldots, \, Z$, а $ \Gamma_1 = Z $ "--- из элементов~$a, \, \ldots, \, z$. Следуя той же логике, получаем:
    \begin{equation}
      \Gamma_3 = \qty{b, \, d, \, f, \, h, \, j, \, l, \, n, \, p, \, r, \, t, \, v, \, x, \, z, \, A, \, B, \, C, \, \ldots, \, Z}.
    \end{equation}
    Количество подмножеств "--- $ 2^{39} $.
  \end{itemize}
  \answer{a) $2^{26}$. \ b) $2^{13}$. \ c) $2^{39}$. }
}


\task{%
  Дана функция~$ m(n) $ натуральной переменной~$n$, которая равна мультимножеству, состоящему из нулей и единиц двоичной записи числа~$n$. К примеру,
  \[
    m(1_{10}) = \qty{1}, \, m(2_{10}) = \qty{1, \, 0}, \, m(10_{10}) = \qty{1, \, 0, \, 1, \, 0}.
  \]
  Существуют ли такие числа~$\alpha$~и~$\beta$, что
  \begin{equation}\label{eq}%
    m(\alpha) = m(\beta), \, \alpha \not= \beta?
  \end{equation}
  Если да, то для любого ли числа~$\alpha$ можно найти такое~$\beta$, что выполняется~\eqref{eq}? Если не для любого, то укажите все такие~$\alpha$, для которых не существует~$\beta$, при котором выполняется~\eqref{eq}.
}

\solution{%
  Такие числа действительно существуют, к примеру:
  \begin{equation}
    \alpha = 5_{10} = 101_{2}, \, \beta = 6_{10} = 110_{2}, \, m(\alpha) = m(\beta) = \qty{1, \, 1, \, 0}.
  \end{equation}
  Объясняется это тем, что, как и в множествах, в мультимножествах порядок следования элементов не имеет значения.

  Но не для любого $ \alpha $ существует соответствующее~$\beta$, например, для единицы такого~$\beta$ не существует, так как любое число, большее единицы, содержит либо один ноль, либо большее количество единиц в двоичной записи.

  Единица "--- не единственное число, для которого не существует соответствующего~$\beta$, подойдёт любое число, в двоичной записи которого нельзя переставить цифры таким образом, чтобы получилось другое число. Под такое описание подходят лишь числа следующего вида:
  \begin{equation}
    \begin{array}{llllll}%
      1_{10} = 1_2,  & \ 3_{10} = 11_2,  & \ 7_{10} = 111_2,  & \ 15_{10} = 1111_2,  & \ 31_{10} = 11111_2,  & \ \ldots \\ 
      2_{10} = 10_2, & \ 4_{10} = 100_2, & \ 8_{10} = 1000_2, & \ 16_{10} = 10000_2, & \ 32_{10} = 100000_2, & \ \ldots \\
    \end{array}
  \end{equation}
  Действительно, если в двоичной записи числа есть как минимум две единицы и ноль, то младшую единицу можно поменять местами с этим нулём, получив другое число.
  \answer{Существуют. Не для любого. $ \qty{1, \, 2, \, 3, \, 4, \, 7, \, 8, \, 15, \, 16, \, 31, \, 32, \, \ldots} $.}
}