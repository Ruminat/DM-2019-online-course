%!TeX root = ../Практика.tex
% Отношения порядка
\task{%
  Дано бинарное отношение~$R$.
  Укажите, является ли оно предпорядком,
  частичным порядком,
  линейным порядком,
  полным линейным порядком.
  Если является линейным порядком, укажите минимальный и максимальный элементы (при наличии).
  \begin{enumerate}[a)]%
    \item $ R = \{ \Vector{a, \, b} \in \R^2 \mid \cos^2 a + \sin^2 b = 1 \} $.
    \item $ R = \{ \Vector{a, \, b} \in \R^2 \mid \sgn (a - b) + 1 \not= 0 \} $.
  \end{enumerate}
}
\solution{%
  Напомним следующие определения:
  \begin{itemize}%
    \item Бинарное отношение~$R$ называется \textbf{предпорядком}, если оно рефлексивно и транзитивно.
    \item Предпорядок называется \textbf{частичным порядком}, если он антисимметричен.
    \item Частичный порядок называется \textbf{линейным},
      если для любых элементов~$a, b \in A$ либо~$ a \leq b $,
      либо~$ b \leq a $.
    \item Линейный порядок на множестве~$A$ называется \textbf{полным},
      если каждое непустое подмножество множества~$A$ имеет наименьший элемент.  
  \end{itemize}
  \begin{enumerate}[a)]%
    \item Ясно, что наше отношение является рефлексивным
      (ввиду основного тригонометрического тождества~$ \cos^2 x + \sin^2 x = 1 $),
      Покажем, что оно является и транзитивным.
      Пусть~$aRb$~и~$bRc$, тогда:
      \[
        \begin{array}{rl}%
            & \cos^2 {\color{Mred} a} + \sin^2 {\color{Mgreen} b} = 1, \\
            & \cos^2 {\color{Mgreen} b} + \sin^2 {\color{Mblue} c} = 1, \\
            & \cos^2 {\color{Mred} a} + \sin^2 {\color{Mblue} c} = \\ 
          = & (1 - \sin^2 b) + (1 - \cos^2 b) = \\ 
          = & 2 - (\cos^2 b + \sin^2 b) = \\ 
          = & 1. \\ 
        \end{array}
      \]
      Значит, из~$aRb$~и~$bRc$ следует~$aRc$.
      Получается, что наше \textit{отношение является предпорядком}.
      Тем не менее, антисимметричным~$R$ не является.
      Действительно, если для любого~$a$ взять~$b = a + 2\pi$,
      то мы получим~$aRb$~и~$bRa$, но в то же время~$a \not= b$.

      Итого, наше отношение не является ни частичным порядком,
      ни линейным порядком,
      ни полным линейным порядком.

    \item Наше отношение, разумеется, является рефлексивным,
      т.\,к. выполняется $aRa \ \forall a \in \R$:
      \[
        \sgn ({\color{gray} \underbrace{\color{black} a - a}_{0}}) + 1 = 1 \not= 0.
      \]
      Является оно и транзитивным. Пусть~$aRb$~и~$bRc$, тогда:
      \[
        \begin{array}{rllll}%
           & \sgn ({\color{Mred} a} - {\color{Mgreen} b}) + 1  & \not= & 0 & \Rightarrow a \geq b, \\
           & \sgn ({\color{Mgreen} b} - {\color{Mblue} c}) + 1 & \not= & 0 & \Rightarrow b \geq c, \\ 
          \Rightarrow & \sgn ({\color{Mred} a} - {\color{Mblue} c}) + 1   & \not= & 0 & \text{ (т.\,к. $a \geq b \geq c$)}.
        \end{array}
      \]
      Это значит, что \textit{отношение является предпорядком}.

      Покажем, что наше отношение является антисимметричным. Пусть $aRb$~и~$bRa$:
      \[
        \begin{array}{rl}%
           & \sgn (a - b) + 1 \not= 0 \text{ и } \sgn (b - a) + 1 \not= 0 \Rightarrow \\ 
          \Rightarrow & a \leq b \text{ и } b \leq a \Rightarrow a = b.
        \end{array}
      \]
      А это значит, что наше \textit{отношение является и частичным порядком}.

      Понятно, что наше \textit{отношение является так же и линейным порядком},
      т.\,к. для любых $a$~и~$b$ из~$\R$ выполняется
      \[
        \text{либо } \sgn (a - b) + 1 \not= 0, \text{ либо } \sgn (b - a) + 1 \not= 0.
      \]

      Тем не менее, \textit{полным линейным порядком наше отношение не является},
      потому что, к примеру, на подмножестве~$(-\infty; \, 0)$ нет наименьшего элемента.
  \end{enumerate}
}
\answer{%
  \begin{enumerate}[a)]%
    \item Является только предпорядком.
    \item Является предпорядком, частичным порядком, линейным порядком, но полным линейным порядком не является.
  \end{enumerate}
}