%!TeX root = ./../Практика.tex
% Бинарные отношения
\task{%
  Даны множества~$A$~и~$B$:
  \[
    A = B = \{\text{{\color{Mred} камень}, {\color{Mgreen} ножницы}, {\color{Mblue} бумага}}\}.
  \]
  Определим бинарное отношение~$R$ следующим образом:
  \[
    R = \text{«побеждает»}\colon \ aRb \iff a \text{ побеждает } b.
  \]
  То есть, к примеру, $ \Vector{\text{{\color{Mred} камень}, {\color{Mgreen} ножницы}}} \in R $,
  но $ \Vector{\text{{\color{Mred} камень}, {\color{Mblue} бумага}}} \not\in R $.
  Выпишите~$R$ в явном виде (множество векторов).
  Найдите:
  \[
    \text{$\delta_R$, $\rho_R$, $R^{-1}$, $-R$, $R(\{\text{{\color{Mred} камень}, {\color{Mblue} бумага}}\})$.}
  \]
}
\solution{%
  Выпишем~$R$ в явном виде:
  \[
    R = \{
      \Vector{\text{{\color{Mred} камень}, {\color{Mgreen} ножницы}}}, \
      \Vector{\text{{\color{Mgreen} ножницы}, {\color{Mblue} бумага}}}, \
      \Vector{\text{{\color{Mblue} бумага}, {\color{Mred} камень}}}
    \}.
  \]
  
  Понятно, что~$\delta_R = \rho_R = \{\text{{\color{Mred} камень}, {\color{Mgreen} ножницы}, {\color{Mblue} бумага}}\}$,
  так как для любого элемента~$a \in A$ найдётся элемент~$b \in A$ (напомним, что~$A = B$) такой, что $ aRb $.
  
  Напомним, что обратное бинарное отношение к~$R$ есть ничто иное, как:
  \begin{equation}
    R^{-1} = \{\Vector{a, \, b} \in A \times B \mid bRa\}.
  \end{equation}
  Выпишем~$R^{-1}$:
  \[
    R^{-1} = \{
      \Vector{\text{{\color{Mgreen} ножницы}, {\color{Mred} камень}}}, \
      \Vector{\text{{\color{Mblue} бумага}, {\color{Mgreen} ножницы}}}, \
      \Vector{\text{{\color{Mred} камень}, {\color{Mblue} бумага}}}
    \}.
  \]

  Напомним, что дополнением к бинарному отношению~$R$ является:
  \begin{equation}
    -R = (A \times B) \setminus R.
  \end{equation}
  Выпишем все векторы~$-R$:
  \[
    \begin{array}{lll}%
      \Vector{\text{{\color{Mred} камень}, {\color{Mblue} бумага}}}, & 
      \Vector{\text{{\color{Mgreen} ножницы}, {\color{Mred} камень}}}, & 
      \Vector{\text{{\color{Mblue} бумага}, {\color{Mgreen} ножницы}}}, \\
      \Vector{\text{{\color{Mred} камень}, {\color{Mred} камень}}}, & 
      \Vector{\text{{\color{Mgreen} ножницы}, {\color{Mgreen} ножницы}}}, & 
      \Vector{\text{{\color{Mblue} бумага}, {\color{Mblue} бумага}}}.
    \end{array}
  \]

  Образом множества~$X$ относительно бинарного отношения~$R$ называется множество:
  \begin{equation}
    R(X) = \{b \in B \mid \exists \, a \in X : aRb\}.
  \end{equation}
  Выпишем образ множества~$ \{\text{{\color{Mred} камень}, {\color{Mblue} бумага}}\} $:
  \[
    R(\{\text{{\color{Mred} камень}, {\color{Mblue} бумага}}\}) = \{
      \Vector{\text{{\color{Mred} камень}, {\color{Mgreen} ножницы}}}, \
      \Vector{\text{{\color{Mblue} бумага}, {\color{Mred} камень}}}
    \}.
  \]
}