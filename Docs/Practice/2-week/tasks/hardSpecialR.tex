%!TeX root = ./../Практика.tex
% Специальные бинарные отношения
\task{%
  Определите, является ли эквивалентностью бинарное отношение~$R$. Если да, то опишите соответствующее фактор-множество.
  \begin{enumerate}[a)]%
    \item $ R = \{ \Vector{a, \, b} \in \N^2 \mid a = b \text{ или } a^2 + b^2 \text{ "--- простое число} \} $.
    \item $ R = \{ \Vector{a, \, b} \in \N^2 \mid a \cdot b \text{ "--- квадрат некоторого числа } n \in \N \} $.
  \end{enumerate}
}
\solution{%
  \begin{enumerate}[a)]%
      \item Довольно очевидно, что наше отношение является рефлексивным и симметричным
        (это понятно из условия «$ a = b $» и ассоциативности сложения).
        Вопрос заключается в том, является ли оно транзитивным.
        Оказывается, не является "--- взять хотя бы такой пример:
        \[
          \begin{array}{rcll}%
            {\color{Mred} 1}^2 + {\color{Mgreen} 2}^2 & = & 5 &  \text{ "--- простое число}, \\ 
            {\color{Mgreen} 2}^2 + {\color{Mblue} 3}^2 & = & 13 & \text{ "--- простое число}, \\ 
            {\color{Mred} 1}^2 + {\color{Mblue} 3}^2 & = & 10 & \text{ "--- составное число}. \\ 
          \end{array}
        \]
        А это значит, что и эквивалентностью~$R$ не является.

      \item Рефлексивность данного отношения, опять же, очевидна
        ($a^2$ "--- всегда квадрат числа~$a$, то есть, всегда выполняется~$aRa$).
        То же касается и симметричности (объясняется это ассоциативностью умножения).

        Проверим, является ли данное отношение транзитивным.
        Запишем числа~$a$,~$b$~и~$c$
        в виде бесконечного произведения простых чисел в некоторых целых степенях:
        \begin{equation}\label{primes}
          \begin{array}{rcl}%
                   {\color{Mred} a}
             & = & 2^{{\color{Mred} x_1}} \cdot
                   3^{{\color{Mred} x_2}} \cdot
                   5^{{\color{Mred} x_3}} \cdot
                   7^{{\color{Mred} x_4}} \cdot \ldots, \\ 
                   {\color{Mgreen} b}
             & = & 2^{{\color{Mgreen} y_1}} \cdot
                   3^{{\color{Mgreen} y_2}} \cdot
                   5^{{\color{Mgreen} y_3}} \cdot
                   7^{{\color{Mgreen} y_4}} \cdot \ldots, \\ 
                   {\color{Mblue} c}
             & = & 2^{{\color{Mblue} z_1}} \cdot
                   3^{{\color{Mblue} z_2}} \cdot
                   5^{{\color{Mblue} z_3}} \cdot
                   7^{{\color{Mblue} z_4}} \cdot \ldots \\   
          \end{array}
        \end{equation}
        Покажем, к примеру, как записать подобным образом числа 1 и 36
        ($ 36 = 2^{{\color{Mred} 2}} \cdot 3^{{\color{Mred} 2}} $):
        \[
          1  = 2^0 \cdot 3^0 \cdot 5^0 \cdot 7^0 \cdot \ldots, \ 
          36 = 2^{{\color{Mred} 2}} \cdot 3^{{\color{Mred} 2}} \cdot 5^0 \cdot 7^0 \cdot \ldots
        \]
        Заметим, что квадратом любого натурального числа~$n$ будет такое число,
        все степени в записи~\eqref{primes} которого будут чётными
        (так как возведение в квадрат числа~$n$ равносильно умножению на 2 степеней простых чисел).

        Вспомним, что при умножении степени складываются.
        Предположим, что~$aRb$~и~$bRc$, тогда получим:
        \[
          \begin{array}{rcll}%
            ({\color{Mred} x_1} + {\color{Mgreen} y_1}) & \text{ и } &
            ({\color{Mgreen} y_1} + {\color{Mblue} z_1}) & \text{ "--- чётные числа}, \\ 
            ({\color{Mred} x_2} + {\color{Mgreen} y_2}) & \text{ и } &
            ({\color{Mgreen} y_2} + {\color{Mblue} z_2}) & \text{ "--- чётные числа}, \\ 
            ({\color{Mred} x_3} + {\color{Mgreen} y_3}) & \text{ и } &
            ({\color{Mgreen} y_3} + {\color{Mblue} z_3}) & \text{ "--- чётные числа}, \\ 
            ({\color{Mred} x_4} + {\color{Mgreen} y_4}) & \text{ и } &
            ({\color{Mgreen} y_4} + {\color{Mblue} z_4}) & \text{ "--- чётные числа}, \\ 
            \vdots & \vdots & \vdots & \\ 
          \end{array}
        \]

        А теперь вспомним арифметику:
        если при сложении двух чисел~$p$~и~$q$ мы получили чётное число~$w$,
        то числа $p$~и~$q$ обязаны быть либо оба чётными, либо оба нечётными.
        Получается, числа~${\color{Mred} x_k}$,~${\color{Mgreen} y_k}$~и~${\color{Mblue} z_k}$
        обязаны иметь одинаковую чётность при любом натуральном~$k$,
        а это в то же время значит, что $ {\color{Mred} x_k} + {\color{Mblue} z_k} $
        всегда будет чётным числом.
        Выходит, что наше отношение является транзитивным, а значит, является и эквивалентностью.

        В таком случае, нам осталось лишь выписать соответствующее фактор-множество.
        Для начала напомним, что классом эквивалентности~$a \in A$ по~$R$ называется множество~$ a / R $:
        \begin{equation}
          a / R = \{b \in B \mid bRa\}.
        \end{equation}
        Фактор-множеством (обозначим его как~$F$) же является множество всех классов эквивалентности.
        Понятно, что для любого числа~$a$, если записать его как в выражении~\eqref{primes},
        $ a / R $ будет состоять из чисел,
        степени простых чисел у которых будут совпадать по кратности
        со степенями соответствующих простых чисел из~$a$.
        Выходит, нам нужно просто перебрать все возможные кратности степеней.
        Таким образом мы получим следующие классы эквивалентности:
        \[
          F_{{\color{Mgreen} i_1}, {\color{Mgreen} i_2}, {\color{Mgreen} i_3}, \ldots} = \qty{
            2^{{\color{Mgreen} i_1} + 2{\color{Mred} k_1}} \cdot
            3^{{\color{Mgreen} i_2} + 2{\color{Mred} k_2}} \cdot
            5^{{\color{Mgreen} i_3} + 2{\color{Mred} k_3}} \cdot \ldots
            \mid
            {\color{Mred} k_1},
            {\color{Mred} k_2},
            {\color{Mred} k_3}, \ldots \in \N_0
          },
        \]
        где каждое из~$ {\color{Mgreen} i_n} $ будет принимать значения 0 и 1
        (что меняет кратность степени).
        Тогда фактор-множество~$F$ будет равно:
        \[
          F = \qty{
            F_{{\color{Mgreen} i_1}, {\color{Mgreen} i_2}, {\color{Mgreen} i_3}, \ldots}
            \mid
            {\color{Mgreen} i_1}, {\color{Mgreen} i_2}, {\color{Mgreen} i_3}, \ldots \in \{0, \, 1\}
          }.
        \]
    \end{enumerate}
}
\answer{%
  \begin{enumerate}[a)]%
    \item Не является эквивалентностью.
    \item Является эквивалентностью. Фактор-множество (F):
      \[
        F = \qty{
          F_{i_1, i_2, i_3, \ldots}
          \mid
          i_1, i_2, i_3, \ldots \in \{0, \, 1\}
        }, \text{ где}
      \]
      \[
        F_{i_1, i_2, i_3, \ldots} = \qty{
          2^{i_1 + 2k_1} \cdot
          3^{i_2 + 2k_2} \cdot
          5^{i_3 + 2k_3} \cdot \ldots
          \mid
          k_1,
          k_2,
          k_3, \ldots \in \N_0
        }.
      \]
  \end{enumerate}
}