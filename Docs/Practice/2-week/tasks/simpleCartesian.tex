%!TeX root = ./../Практика.tex
% Декартово произведение
\task{%
  Выпишите все элементы множества~$A$.
  \begin{enumerate}[a)]%
    \item $ A = \qty{1, \, 2, \, 3} \times \qty{a, \, b, \, c} $.
    \item $ A = \qty(\N \cap \qty[e^2; \ \pi^2]) \times \R^{42} \times \qty(B \divisionsymbol B) $,
      где $B$ "--- некоторое множество.
    \item $ A
      = \qty(X \times \qty(Y \times Z)) \setminus 
        \qty(\qty(X \times Y) \times Z) $, где
        \[
          X = \qty{1, \, 2, \, 3}, \, Y = \qty{\pi, \, e}, \, Z = \qty{0}.
        \]
  \end{enumerate}
}
\solution{%
  \begin{enumerate}[a)]%
    \item Чтобы выписать все элементы декартового произведения~$ X \times Y $,
      можно следовать такому алгоритму:
      каждому элементу множества~$X$ сопоставить каждый элемент множества~$Y$.
      Таким образом декартово произведение
      $ \qty{{\color{Mgreen} 1}, \, {\color{Mgreen} 2}, \, {\color{Mgreen} 3}} \times
        \qty{{\color{Mred} a}, \, {\color{Mred} b}, \, {\color{Mred} c}} $
      даст нам следующие 9~векторов:
      \[
        \begin{array}{cccr}%
          \Vector{{\color{Mgreen} 1}, \, {\color{Mred} a}}, &
          \Vector{{\color{Mgreen} 1}, \, {\color{Mred} b}}, &
          \Vector{{\color{Mgreen} 1}, \, {\color{Mred} c}}, &
          \\ 
          \Vector{{\color{Mgreen} 2}, \, {\color{Mred} a}}, &
          \Vector{{\color{Mgreen} 2}, \, {\color{Mred} b}}, &
          \Vector{{\color{Mgreen} 2}, \, {\color{Mred} c}}, &
          \\ 
          \Vector{{\color{Mgreen} 3}, \, {\color{Mred} a}}, &
          \Vector{{\color{Mgreen} 3}, \, {\color{Mred} b}}, &
          \Vector{{\color{Mgreen} 3}, \, {\color{Mred} c}}. &
          \\ 
        \end{array}
      \]
      Заметим, что декартово произведение
      $ \qty{{\color{Mred} a}, \, {\color{Mred} b}, \, {\color{Mred} c}} \times
        \qty{{\color{Mgreen} 1}, \, {\color{Mgreen} 2}, \, {\color{Mgreen} 3}} $
      дало бы нам те же самые векторы с переставленными элементами
      (то есть, к примеру, вектор
       $ \Vector{{\color{Mgreen} 1}, \, {\color{Mred} a}} $
       превратился бы в~$ \Vector{{\color{Mred} a}, \, {\color{Mgreen} 1}} $),
      а это, вообще говоря, уже другие векторы (порядок элементов в векторе имеет значение).
      Иными словами, \textit{в общем случае} декартово произведение не коммутативно:
      \begin{equation}
        A \times B \not= B \times A.
      \end{equation}
      
    \item Прежде чем что-то выписывать, заметим, что~$ B \divisionsymbol B = \varnothing $. Декартово произведение любого множества с пустым множеством даст нам пустое множество. Таким образом, мы получаем~$ A = \varnothing $.
    
    \item Выпишем сначала векторы, получающиеся в результате~$ X \times \qty(Y \times Z) $:
      \begin{equation}\label{rightCartesian}%
        \begin{array}{cccr}%
          \Vector{{\color{Mgreen} 1}, \, \Vector{{\color{Mred} \pi}, \, 0}}, &
          \Vector{{\color{Mgreen} 2}, \, \Vector{{\color{Mred} \pi}, \, 0}}, &
          \Vector{{\color{Mgreen} 3}, \, \Vector{{\color{Mred} \pi}, \, 0}}, & \\
          \Vector{{\color{Mgreen} 1}, \, \Vector{{\color{Mred} e}, \, 0}},   &
          \Vector{{\color{Mgreen} 2}, \, \Vector{{\color{Mred} e}, \, 0}},   &
          \Vector{{\color{Mgreen} 3}, \, \Vector{{\color{Mred} e}, \, 0}}.   & \\
        \end{array}
      \end{equation}
      А теперь векторы, получающиеся в результате~$ \qty(X \times Y) \times Z $:
      \begin{equation}\label{leftCartesian}
        \begin{array}{cccr}%
          \Vector{\Vector{{\color{Mgreen} 1}, \, {\color{Mred} \pi}}, \, 0}, &
          \Vector{\Vector{{\color{Mgreen} 2}, \, {\color{Mred} \pi}}, \, 0}, &
          \Vector{\Vector{{\color{Mgreen} 3}, \, {\color{Mred} \pi}}, \, 0}, & \\ 
          \Vector{\Vector{{\color{Mgreen} 1}, \, {\color{Mred} e}},   \, 0}, &
          \Vector{\Vector{{\color{Mgreen} 2}, \, {\color{Mred} e}},   \, 0}, &
          \Vector{\Vector{{\color{Mgreen} 3}, \, {\color{Mred} e}},   \, 0}. & \\ 
        \end{array}
      \end{equation}
      Как можно заметить, хоть элементы векторов и совпадают, структуры они имеют разные. В первом случае мы имеем дело с векторами вида~$\Vector{a, \, \Vector{b, \, c}}$, во втором же "--- с векторами вида~$ \Vector{\Vector{a, \, b}, \, c} $. Строго говоря, \textit{в общем случае} декартово произведение не ассоциативно, то есть:
      \begin{equation}
        (A \times B) \times C \not= A \times (B \times C).
      \end{equation}
      Тем не менее, если в выражении $ A \times B \times C $ нет скобок, его можно рассматривать как декартово произведение множеств:
      \begin{equation}
        A \times B \times C = \{ \Vector{a, \, b, \, c} \mid a \in A, \, b \in B, \, c \in C \}.
      \end{equation}
      % Однако довольно часто гораздо удобнее рассматривать векторы~$\Vector{a, \, \Vector{b, \, c}}$~и~$ \Vector{\Vector{a, \, b}, \, c} $ как альтернативные представления вектора~$\Vector{a, \, b, \, c}$, то есть принимать декартово произведение множеств~$A, \, B, \, C$ за
      % \begin{equation}\label{easyPeasy}
      %   A \times B \times C = \{ \Vector{a, \, b, \, c} \mid a \in A, \, b \in B, \, c \in C \}.
      % \end{equation}
      % В данном курсе мы будем придерживаться формального (строгого) определения, если заранее не будет оговорено, что мы имеем дело с определением вида~\eqref{easyPeasy}.

      Однако в нашем примере скобки имеются, поэтому множества:
      \[
        X \times \qty(Y \times Z) \text{ и } \qty(X \times Y) \times Z, \text{ "---}
      \]
      не имеют общих элементов, а это в свою очередь значит, что в ответе мы получим векторы из выражения~\eqref{rightCartesian}. 
  \end{enumerate}
}
\answer{%
  \begin{enumerate}[a)]%
    \item $
        \Vector{1, \, a}, \, \Vector{1, \, b}, \, \Vector{1, \, c}, \
        \Vector{2, \, a}, \, \Vector{2, \, b}, \, \Vector{2, \, c}, \
        \Vector{3, \, a}, \, \Vector{3, \, b}, \, \Vector{3, \, c}
      $.
    \item Нет элементов.
    \item $ \Vector{1, \, \Vector{\pi, \, 0}}, \Vector{2, \, \Vector{\pi, \, 0}}, \Vector{3, \, \Vector{\pi, \, 0}}, \Vector{1, \, \Vector{e, \, 0}}, \Vector{2, \, \Vector{e, \, 0}}, \Vector{3, \, \Vector{e, \, 0}} $.
  \end{enumerate}
}