%!TeX root = ./../Практика.tex
% Функции
\task{%
  Для заданной функции~$ f(x)\colon \R \to \R $
  определите, является ли она инъекцией, сюръекцией, биекцией.
  \begin{enumerate}[a)]%
    \item $ f(x) = \sin [x] + \cos [x] $.
      Здесь под $ [x] $ понимается взятие целой части числа~$x$
      (то есть, к примеру,
      $ [{\color{Mgreen} 5}{\color{Mred} .999}] = {\color{Mgreen} 5}, \
        [{\color{Mgreen} -98}{\color{Mred} .1}] = {\color{Mgreen} -98} $).
    \item $ f(x) = \sgn (x) \cdot x^2 $.
      Здесь под $ \sgn (x) $ понимается функция знака
      (то есть, к примеру, $ \sgn (-7) = -1, \ \sgn (0) = 0, \ \sgn (1729) = 1 $).
  \end{enumerate}
}
\solution{%
  \begin{enumerate}[a)]%
    \item Посмотрим на график функции~$f(x)$:
      \imgt[0.75]{График функции~$ f(x) = \sin [x] + \cos [x] $}{SinPlusCosPlot.pdf}
      По графику видно, что функция ни инъекцией, ни сюръекцией не является.
      Действительно, если взять 2 различных $x$: $0$ и $0.1$, "---
      мы получим одно и тоже значение функции,
      так как целые части аргументов равны:
      \[
        [0] = [0.1] = 0 \Rightarrow
          \sin {\color{gray} \underbrace{\color{black} [0]}_{=0}}
        + \cos {\color{gray} \underbrace{\color{black} [0]}_{=0}}
        = \sin {\color{gray} \underbrace{\color{black} [0.1]}_{=0}}
        + \cos {\color{gray} \underbrace{\color{black} [0.1]}_{=0}}.
      \]
      Поэтому функция~$f(x)$ не является инъекцией.
      Сюръекцией она тоже не является, так как синус и косинус "--- ограниченные функции
      (сумма синуса и косинуса не может превышать число 2),
      то есть, область значений функции~$f(x)$ не равна~$\R$.

    \item Посмотрим на график функции~$f(x)$:
      \imgt[0.75]{График функции~$ f(x) = \sgn (x) \cdot x^2 $}{sgnXx2.pdf}
      По графику, опять же, видно, что функция~$f(x)$ является и инъекцией, и сюръекцией, а значит и биекцией.
      И правда, рассмотрим сначала интервал~$ x \in [0; \, \infty) $.
      Естественно, функция на нём инъективна:
      \[
        x \geq 0 \Rightarrow \text{ при  } x_1 \not= x_2 \quad x_1^2 \not= x_2^2.
      \]
      И, что довольно очевидно, область значений функции~$ f(x) $ на этом интервале
      есть ничто иное, как~$[0; \, \infty)$.
      При~$ x \in (-\infty; \, 0] $ ситуация совершенно идентична,
      с той лишь разницей, что область значений функции равна~$(-\infty; \, 0]$.
      
      Итого, получаем следующее: функция инъективна при $ x \in \R $,
      при этом область значений функции равна~$ \R $,
      иными словами, функция является сюръекцией, и биекцией (так как является и инъекцией, и сюръекцией).
  \end{enumerate}
}
\answer{%
  \begin{enumerate}[a)]%
    \item Не является ни инъекцией, ни сюръекцией, ни биекцией.
    \item Является и инъекцией, и сюръекцией, и биекцией.
  \end{enumerate}
}