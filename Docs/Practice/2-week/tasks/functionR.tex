%!TeX root = ../Практика.tex
% Функции
\task{%
  Дано бинарное отношение~$f$:
  \begin{equation}\label{fQ}%
    f = \{
      \Vector{x, \, y} \in \R^2 \mid x \in \Q \text{ и } |x| = y
      \text{ или } x \not\in \Q \text{ и } |x| = -y
    \}.
  \end{equation}
  Определите, является ли~$f$ функцией.
  Если да, укажите~$\delta_f$,~$\rho_f$, $f^{-1}$ (при наличии);
  определите, является ли~$f$ инъекцией, сюръекцией, биекцией. 
}
\solution{%
  Напомним, что \textbf{функцией} является такое бинарное отношение~$f \in A \times B$,
  у которого~$\delta_f = A$, $\rho_f \subseteq B$, а так же
  \begin{equation}\label{uniqueY}
    \text{для всех~$x \in \delta_f$, } y_1, y_2 \in \rho_f
    \text{ из $x f y_1$ и $x f y_2$ следует } y_1 = y_2.
  \end{equation}

  Наше бинарное отношение и вправду является функцией,
  действительно, любой~$x \in \R$ либо принадлежит~$\Q$, либо не принадлежит,
  при этом что в первом, что во втором случае найдётся такой~$y \in \R$, что
  $ |x| = y \text{ или же } |x| = -y $.
  Это значит, что~$ \delta_f = A = \R $.
  Нет никаких сомнений, что $ \rho_f \subseteq B = \R $.

  Покажем, что выполняется~\eqref{uniqueY}.
  Пусть $ x f y_1 $ и $ x f y_2 $, тогда, если~$x \in \Q$:
  \[
    \qty(|x| = y_1, \ |x| = y_2) \Rightarrow y_1 = y_2.
  \]
  Если~$ x \not\in \Q $:
  \[
    \qty(|x| = -y_1 \Leftrightarrow y_1 = -|x|, \ |x| = -y_2 \Leftrightarrow y_2 = -|x|)
    \Rightarrow y_1 = y_2.
  \]
  Получается, $f$ действительно является функцией.
  График её выглядит примерно следующим образом:
  \imgl[0.45]{График функции~$f$}{Qplot.pdf}{Qplot}
  Может показаться, что область значений функции ($ \rho_f $, синяя область на графике) равна~$\R$,
  однако это не так "--- когда~$x$ является рациональным числом ($ x \in \Q $),
  $y$~принимает неотрицательные значения ($|x|$), в обратном же случае "--- отрицательные ($-|x|$).
  Иными словами, положительные значения функции рациональны, отрицательные же иррациональны.
  Получается
  \[
    \rho_f = \{y \in \Q \mid y \geq 0\} \cup \{y \in \R \setminus \Q \mid y < 0\}.
  \]
  Это в свою очередь говорит о том, что сюръекцией функция~$f$ не является.
  Более того, не является она и инъекцией, так как при значениях~$x = 1$~и~$x = -1$
  значения функции совпадают, что говорит о том, что обратной функции~($ f^{-1} $) не существует.
  А значит, не является она и биекцией.

  Может показаться очевидным, что обратной функции ($ f^{-1} $) не существует,
  это действительно так.
  Тем не менее не всё так просто "--- если в~\eqref{fQ} лишь слегка поменять условие,
  убрав модули у~$x$, мы получим уже совершенно иную функцию, график которой визуально
  совпадает с графиком на Рис.~\ref{Qplot}, но, как ни странно,
  новая функция будет являться и инъекцией, и сюръекцией, и биекцией.
  А это значит, что у неё будет и обратная функция, которая, к тому же, будет равна исходной.
}
\answer{%
  $f$ является функцией, $\delta_f = \R$,
  $ \rho_f = \{y \in \Q \mid y \geq 0\} \cup \{y \in \R \setminus \Q \mid y < 0\} $.
  Обратной функции не существует, не является ни инъекцией, ни сюръекцией, ни биекцией.
}