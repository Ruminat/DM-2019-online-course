%!TeX root = ./../Практика.tex
% Cпециальыне бинарные отношения
\task{%
  Задано бинарное отношение~$R$:
  \begin{equation}
    R = \qty{\Vector{x, \, y} \mid x, y \in \N \text{ и } x \text{ входит в запись числа } y}.
  \end{equation}
  То есть, к примеру,
  $ \Vector{{\color{Mgreen} 8}, \, {12\color{Mgreen} 8}} \in R $,
  $ \Vector{{\color{Mgreen} 10}, \, {\color{Mgreen} 10}10} \in R $,
  но $ \Vector{4, \, 123} \not\in R $,
  $ \Vector{0, \, 28} \not\in R $.
  Определите, является ли отношение~$R$: рефлексивным, иррефлексивным, симметричным, антисимметричным, транзитивным, эквивалентностью.
  % Найдите $ R^{-1} $, $ -R $, $ \delta_R $, $ \rho_R $.
}
% \footnotetext{В данной задаче не рассматриваются числа с ведущими нулями, к примеру, $ 1 = 01 = 001 = \ldots $}
\solution{%
  Бинарное отношение~$R$ называется рефлексивным, если для любого~$a \in A$ выполняется~$aRa$.
  Довольно очевидно, что любое число входит в собственную же запись, поэтому~\textit{$R$ является рефлексивным}.
  
  Иррефлексивным же отношение~$R$ называется, если при любом~$a \in A$ не выполняется~$ aRa $, что, разумеется, не так, следовательно, \textit{иррефлексивным отношение~$R$ не является}.

  Бинарное отношение~$R$ называется симметричным, если:
  \[
    \text{для всех~$a, b \in A$ } \ aRb \Rightarrow bRa.
  \]
  Понятно, что наше \textit{отношение симметричным не является}.
  Взять хотя бы, к примеру, такой вектор:
  $ \Vector{{\color{Mgreen} 13}, \, {\color{Mgreen} 13}37} \in R $, "---
  если переставить местами его элементы, получим:
  $ \Vector{1337, \, 13} \not\in R $.

  Антисимметричным является такое бинарное отношение~$R$, что:
  \[
    \text{для всех $ a, b \in A $ } \ aRb \text{ и } bRa \Rightarrow a = b.
  \]
  Наше \textit{отношение~$R$ является антисимметричным}. Действительно, пусть $ aRb $ и $ bRa $, запишем цифры числа~$a$ как~$a_1, \, \ldots, \, a_n$, цифры числа~$b$ "--- как~$b_1, \, \ldots, \, b_n$ (в~$a$~и~$b$ одинаковое количество цифр по той причине, что число с б$\acute{о}$льшим количеством цифр не может входить в число с меньшим количеством цифр), в таком случае~$a$~и~$b$ можно записать следующим образом:
  \[
    a = {\color{gray} \underbrace{\color{black} b_1\ldots b_n}_{b}}, \
    b = {\color{gray} \underbrace{\color{black} a_1\ldots a_n}_{a}}.
  \]
  Иными словами, $ a = b $.

  Бинарное отношение~$R$ называется транзитивным, если:
  \[
    \text{для всех $ a, b, c \in A $} \ aRb \text{ и } bRc \Rightarrow aRc.
  \]
  Разумеется, наше \textit{отношение является транзитивным}, так как если в записи числа~$c$ содержится~$b$, в записи которого, в свою очередь, содержится~$a$, то~$a$ содержится и в записи~$c$. Для примера возьмём три числа:
  \[
    a = {\color{Mgreen} 2,} \
    b = {\color{Mred} 1}{\color{Mgreen} 2}{\color{Mred} 4}, \
    c = 0{\color{Mred} 1}{\color{Mgreen} 2}{\color{Mred} 4}8.
  \]

  Рефлексивное, симметричное, транзитивное бинарное отношение на~$A$ называется эквивалентностью на~$A$. Наше отношение не является симметричным, а значит и \textit{эквивалентностью оно не является}.
}
\answer{%
  $R$ является рефлексивным, антисимметричным, транзитивным. Не является иррефлексивным, симметричным, эквивалентностью.
}