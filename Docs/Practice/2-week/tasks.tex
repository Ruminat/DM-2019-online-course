%!TeX root = ./Практика.tex

% Декартово произведение
\task{%
  Выпишите все элементы множества~$A$.
  \begin{enumerate}[a)]%
    \item $ A = \qty{1, \, 2, \, 3} \times \qty{a, \, b, \, c} $.
    \item $ A = \qty(\N \cap \qty[e^2; \ \pi^2]) \times \R^{42} \times \qty(B \divisionsymbol B) $,
      где $B$ "--- некоторое множество.
    \item $ A
      = \qty(X \times \qty(Y \times Z)) \setminus 
        \qty(\qty(X \times Y) \times Z) $, где
        \[
          X = \qty{1, \, 2, \, 3}, \, Y = \qty{\pi, \, e}, \, Z = \qty{0}.
        \]
  \end{enumerate}
}
\solution{%
  \begin{enumerate}[a)]%
    \item Чтобы выписать все элементы декартового произведения~$ X \times Y $, можно следовать такому алгоритму: каждому элементу множества~$X$ сопоставить каждый элемент множества~$Y$. Таким образом получим следующие 9~векторов:
      \[
        \begin{array}{cccr}%
          \Vector{{\color{Mgreen} 1}, \, {\color{Mred} a}}, &
          \Vector{{\color{Mgreen} 1}, \, {\color{Mred} b}}, &
          \Vector{{\color{Mgreen} 1}, \, {\color{Mred} c}}, &
          \\ 
          \Vector{{\color{Mgreen} 2}, \, {\color{Mred} a}}, &
          \Vector{{\color{Mgreen} 2}, \, {\color{Mred} b}}, &
          \Vector{{\color{Mgreen} 2}, \, {\color{Mred} c}}, &
          \\ 
          \Vector{{\color{Mgreen} 3}, \, {\color{Mred} a}}, &
          \Vector{{\color{Mgreen} 3}, \, {\color{Mred} b}}, &
          \Vector{{\color{Mgreen} 3}, \, {\color{Mred} c}}. &
          \\ 
        \end{array}
      \]
      
    \item Прежде чем что-то выписывать, заметим, что~$ B \divisionsymbol B = \varnothing $. Декартово произведение любого множества с пустым множеством даст нам пустое множество. Таким образом, мы получаем~$ A = \varnothing $.
    
    \item Выпишем сначала векторы, получающиеся в результате~$ X \times \qty(Y \times Z) $:
      \begin{equation}\label{rightCartesian}%
        \begin{array}{cccr}%
          \Vector{{\color{Mgreen} 1}, \, \Vector{{\color{Mred} \pi}, \, 0}}, &
          \Vector{{\color{Mgreen} 2}, \, \Vector{{\color{Mred} \pi}, \, 0}}, &
          \Vector{{\color{Mgreen} 3}, \, \Vector{{\color{Mred} \pi}, \, 0}}, & \\
          \Vector{{\color{Mgreen} 1}, \, \Vector{{\color{Mred} e}, \, 0}},   &
          \Vector{{\color{Mgreen} 2}, \, \Vector{{\color{Mred} e}, \, 0}},   &
          \Vector{{\color{Mgreen} 3}, \, \Vector{{\color{Mred} e}, \, 0}}.   & \\
        \end{array}
      \end{equation}
      А теперь векторы, получающиеся в результате~$ \qty(X \times Y) \times Z $:
      \begin{equation}\label{leftCartesian}
        \begin{array}{cccr}%
          \Vector{\Vector{{\color{Mgreen} 1}, \, {\color{Mred} \pi}}, \, 0}, &
          \Vector{\Vector{{\color{Mgreen} 2}, \, {\color{Mred} \pi}}, \, 0}, &
          \Vector{\Vector{{\color{Mgreen} 3}, \, {\color{Mred} \pi}}, \, 0}, & \\ 
          \Vector{\Vector{{\color{Mgreen} 1}, \, {\color{Mred} e}},   \, 0}, &
          \Vector{\Vector{{\color{Mgreen} 2}, \, {\color{Mred} e}},   \, 0}, &
          \Vector{\Vector{{\color{Mgreen} 3}, \, {\color{Mred} e}},   \, 0}. & \\ 
        \end{array}
      \end{equation}
      Как можно заметить, хоть элементы векторов и совпадают, структуры они имеют разные. В первом случае мы имеем дело с векторами вида~$\Vector{a, \, \Vector{b, \, c}}$, во втором же "--- с векторами вида~$ \Vector{\Vector{a, \, b}, \, c} $. Строго говоря, в общем случае декартово произведение не ассоциативно, то есть:
      \begin{equation}
        (A \times B) \times C \not= A \times (B \times C).
      \end{equation}
      Тем не менее, если в выражении $ A \times B \times C $ нет скобок, его можно рассматривать как декартово произведение множеств:
      \begin{equation}
        A \times B \times C = \{ \Vector{a, \, b, \, c} \mid a \in A, \, b \in B, \, c \in C \}.
      \end{equation}
      % Однако довольно часто гораздо удобнее рассматривать векторы~$\Vector{a, \, \Vector{b, \, c}}$~и~$ \Vector{\Vector{a, \, b}, \, c} $ как альтернативные представления вектора~$\Vector{a, \, b, \, c}$, то есть принимать декартово произведение множеств~$A, \, B, \, C$ за
      % \begin{equation}\label{easyPeasy}
      %   A \times B \times C = \{ \Vector{a, \, b, \, c} \mid a \in A, \, b \in B, \, c \in C \}.
      % \end{equation}
      % В данном курсе мы будем придерживаться формального (строгого) определения, если заранее не будет оговорено, что мы имеем дело с определением вида~\eqref{easyPeasy}.

      В результате мы получаем лишь векторы из~$ X \times \qty(Y \times Z) $.
  \end{enumerate}
}
\answer{%
  \begin{enumerate}[a)]%
    \item $
        \Vector{1, \, a}, \, \Vector{1, \, b}, \, \Vector{1, \, c}, \
        \Vector{2, \, a}, \, \Vector{2, \, b}, \, \Vector{2, \, c}, \
        \Vector{3, \, a}, \, \Vector{3, \, b}, \, \Vector{3, \, c}
      $.
    \item Нет элементов.
    \item $ \Vector{1, \, \Vector{\pi, \, 0}}, \Vector{2, \, \Vector{\pi, \, 0}}, \Vector{3, \, \Vector{\pi, \, 0}}, \Vector{1, \, \Vector{e, \, 0}}, \Vector{2, \, \Vector{e, \, 0}}, \Vector{3, \, \Vector{e, \, 0}} $.
  \end{enumerate}
}

% Декартово произведение
\task{%
  Дано множество~$ \mathbb B = \{0, \, 1\} $. Для каждого натурального~$n$ найдите сумму количеств единиц в каждом из векторов множества~$\mathbb B^n$. К примеру, при~$n = 2$:
  \[
    \mathbb B^2 = \{\Vector{0, \, 0}, \, \Vector{0, \, 1}, \, \Vector{1, \, 0}, \, \Vector{1, \, 1}\},
  \]
  в сумме получаем 4 единицы.
}
\solution{%
  % Выпишем суммы для первых 3-х~$n$:
  % \[
  %   \begin{array}{rclrl}%
  %     n = 1 & \Rightarrow & \mathbb B   & \Rightarrow & 1 \text{ единица}, \\ 
  %     n = 2 & \Rightarrow & \mathbb B^2 & \Rightarrow & 4 \text{ единицы}, \\ 
  %     n = 3 & \Rightarrow & \mathbb B^3 & \Rightarrow & 12 \text{ единиц}. \\ 
  %   \end{array}
  % \]
  Заметим, что общее количество нулей и единиц для конкретного~$n$ есть ничто иное, как $ n \cdot 2^n $.
  Действительно, множество~$\mathbb B^n$ содержит $2^n$ векторов, каждый из которых состоит из $n$ элементов (нулей и единиц). К примеру, для~$ n = 3 $ имеем (8~векторов по 3~элемента в каждом):
  \[
    \begin{array}{c}%
      \underset{(1)}{\Vector{0, \, 0, \, 0}}, \,
      \underset{(2)}{\Vector{0, \, 0, \, 1}}, \,
      \underset{(3)}{\Vector{0, \, 1, \, 0}}, \,
      \underset{(4)}{\Vector{0, \, 1, \, 1}}, \\ 
      \underset{(5)}{\Vector{1, \, 0, \, 0}}, \,
      \underset{(6)}{\Vector{1, \, 0, \, 1}}, \,
      \underset{(7)}{\Vector{1, \, 1, \, 0}}, \,
      \underset{(8)}{\Vector{1, \, 1, \, 1}}. \\ 
    \end{array}
  \]
  Заметим, что в сумме для всех векторов количество нулей и единиц совпадает. И вправду, для~$n = 1$ это очевидно, а для каждого последующего~$n$ мы получаем векторы путём добавления нуля и единицы к каждому текущему вектору, то есть, к примеру, из $\mathbb B^2$ получаем следующие векторы для $\mathbb B^3$:
  \[
    \begin{array}{c}%
      \Vector{{\color{Mgreen} 0}, \, {\color{Mgreen} 0}} \to
      \Vector{{\color{Mgreen} 0}, \, {\color{Mgreen} 0}, \, {\color{Mred} 0}}, \,
      \Vector{{\color{Mgreen} 0}, \, {\color{Mgreen} 0}, \, {\color{Mred} 1}}, \\ 
      \Vector{{\color{Mgreen} 0}, \, {\color{Mgreen} 1}} \to
      \Vector{{\color{Mgreen} 0}, \, {\color{Mgreen} 1}, \, {\color{Mred} 0}}, \,
      \Vector{{\color{Mgreen} 0}, \, {\color{Mgreen} 1}, \, {\color{Mred} 1}}, \\ 
      \Vector{{\color{Mgreen} 1}, \, {\color{Mgreen} 0}} \to
      \Vector{{\color{Mgreen} 1}, \, {\color{Mgreen} 0}, \, {\color{Mred} 0}}, \,
      \Vector{{\color{Mgreen} 1}, \, {\color{Mgreen} 0}, \, {\color{Mred} 1}}, \\ 
      \Vector{{\color{Mgreen} 1}, \, {\color{Mgreen} 1}} \to
      \Vector{{\color{Mgreen} 1}, \, {\color{Mgreen} 1}, \, {\color{Mred} 0}}, \,
      \Vector{{\color{Mgreen} 1}, \, {\color{Mgreen} 1}, \, {\color{Mred} 1}}. \\ 
    \end{array}
  \]
  Значит, в результате мы получаем $ n \cdot 2^{n - 1} $ единиц.
}
\answer{%
  $ n \cdot 2^{n - 1} $.
}

% Функции
\task{%
  Задана функция~$ f $:
  \begin{equation}
    f \colon \R \to \R, \ f(x) = \sgn (x) \cdot \qty([x]^2 + \{x\}^2),
  \end{equation}
  Найдите область определения и множество значений функции~$f$. Определите, является ли функция инъекцией, сюръекцией, биекцией.
}

% \task{%
%   somebody once told me the world is gonna roll me
% }