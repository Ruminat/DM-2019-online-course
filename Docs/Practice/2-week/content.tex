%!TeX root = ./Практика.tex
\TITLE{Неделя 2. Практическое занятие}{Отношения и функции}


\subtitle{Разбор задач}

%!TeX root = ./../Практика.tex
% Декартово произведение
\task{%
  Выпишите все элементы множества~$A$.
  \begin{enumerate}[a)]%
    \item $ A = \qty{1, \, 2, \, 3} \times \qty{a, \, b, \, c} $.
    \item $ A = \qty(\N \cap \qty[e^2; \ \pi^2]) \times \R^{42} \times \qty(B \divisionsymbol B) $,
      где $B$ "--- некоторое множество.
    \item $ A
      = \qty(X \times \qty(Y \times Z)) \setminus 
        \qty(\qty(X \times Y) \times Z) $, где
        \[
          X = \qty{1, \, 2, \, 3}, \, Y = \qty{\pi, \, e}, \, Z = \qty{0}.
        \]
  \end{enumerate}
}
\solution{%
  \begin{enumerate}[a)]%
    \item Чтобы выписать все элементы декартового произведения~$ X \times Y $,
      можно следовать такому алгоритму:
      каждому элементу множества~$X$ сопоставить каждый элемент множества~$Y$.
      Таким образом декартово произведение
      $ \qty{{\color{Mgreen} 1}, \, {\color{Mgreen} 2}, \, {\color{Mgreen} 3}} \times
        \qty{{\color{Mred} a}, \, {\color{Mred} b}, \, {\color{Mred} c}} $
      даст нам следующие 9~векторов:
      \[
        \begin{array}{cccr}%
          \Vector{{\color{Mgreen} 1}, \, {\color{Mred} a}}, &
          \Vector{{\color{Mgreen} 1}, \, {\color{Mred} b}}, &
          \Vector{{\color{Mgreen} 1}, \, {\color{Mred} c}}, &
          \\ 
          \Vector{{\color{Mgreen} 2}, \, {\color{Mred} a}}, &
          \Vector{{\color{Mgreen} 2}, \, {\color{Mred} b}}, &
          \Vector{{\color{Mgreen} 2}, \, {\color{Mred} c}}, &
          \\ 
          \Vector{{\color{Mgreen} 3}, \, {\color{Mred} a}}, &
          \Vector{{\color{Mgreen} 3}, \, {\color{Mred} b}}, &
          \Vector{{\color{Mgreen} 3}, \, {\color{Mred} c}}. &
          \\ 
        \end{array}
      \]
      Заметим, что декартово произведение
      $ \qty{{\color{Mred} a}, \, {\color{Mred} b}, \, {\color{Mred} c}} \times
        \qty{{\color{Mgreen} 1}, \, {\color{Mgreen} 2}, \, {\color{Mgreen} 3}} $
      дало бы нам те же самые векторы с переставленными элементами
      (то есть, к примеру, вектор
       $ \Vector{{\color{Mgreen} 1}, \, {\color{Mred} a}} $
       превратился бы в~$ \Vector{{\color{Mred} a}, \, {\color{Mgreen} 1}} $),
      а это, вообще говоря, уже другие векторы (порядок элементов в векторе имеет значение).
      Иными словами, \textit{в общем случае} декартово произведение не коммутативно:
      \begin{equation}
        A \times B \not= B \times A.
      \end{equation}
      
    \item Прежде чем что-то выписывать, заметим, что~$ B \divisionsymbol B = \varnothing $. Декартово произведение любого множества с пустым множеством даст нам пустое множество. Таким образом, мы получаем~$ A = \varnothing $.
    
    \item Выпишем сначала векторы, получающиеся в результате~$ X \times \qty(Y \times Z) $:
      \begin{equation}\label{rightCartesian}%
        \begin{array}{cccr}%
          \Vector{{\color{Mgreen} 1}, \, \Vector{{\color{Mred} \pi}, \, 0}}, &
          \Vector{{\color{Mgreen} 2}, \, \Vector{{\color{Mred} \pi}, \, 0}}, &
          \Vector{{\color{Mgreen} 3}, \, \Vector{{\color{Mred} \pi}, \, 0}}, & \\
          \Vector{{\color{Mgreen} 1}, \, \Vector{{\color{Mred} e}, \, 0}},   &
          \Vector{{\color{Mgreen} 2}, \, \Vector{{\color{Mred} e}, \, 0}},   &
          \Vector{{\color{Mgreen} 3}, \, \Vector{{\color{Mred} e}, \, 0}}.   & \\
        \end{array}
      \end{equation}
      А теперь векторы, получающиеся в результате~$ \qty(X \times Y) \times Z $:
      \begin{equation}\label{leftCartesian}
        \begin{array}{cccr}%
          \Vector{\Vector{{\color{Mgreen} 1}, \, {\color{Mred} \pi}}, \, 0}, &
          \Vector{\Vector{{\color{Mgreen} 2}, \, {\color{Mred} \pi}}, \, 0}, &
          \Vector{\Vector{{\color{Mgreen} 3}, \, {\color{Mred} \pi}}, \, 0}, & \\ 
          \Vector{\Vector{{\color{Mgreen} 1}, \, {\color{Mred} e}},   \, 0}, &
          \Vector{\Vector{{\color{Mgreen} 2}, \, {\color{Mred} e}},   \, 0}, &
          \Vector{\Vector{{\color{Mgreen} 3}, \, {\color{Mred} e}},   \, 0}. & \\ 
        \end{array}
      \end{equation}
      Как можно заметить, хоть элементы векторов и совпадают, структуры они имеют разные. В первом случае мы имеем дело с векторами вида~$\Vector{a, \, \Vector{b, \, c}}$, во втором же "--- с векторами вида~$ \Vector{\Vector{a, \, b}, \, c} $. Строго говоря, \textit{в общем случае} декартово произведение не ассоциативно, то есть:
      \begin{equation}
        (A \times B) \times C \not= A \times (B \times C).
      \end{equation}
      Тем не менее, если в выражении $ A \times B \times C $ нет скобок, его можно рассматривать как декартово произведение множеств:
      \begin{equation}
        A \times B \times C = \{ \Vector{a, \, b, \, c} \mid a \in A, \, b \in B, \, c \in C \}.
      \end{equation}
      % Однако довольно часто гораздо удобнее рассматривать векторы~$\Vector{a, \, \Vector{b, \, c}}$~и~$ \Vector{\Vector{a, \, b}, \, c} $ как альтернативные представления вектора~$\Vector{a, \, b, \, c}$, то есть принимать декартово произведение множеств~$A, \, B, \, C$ за
      % \begin{equation}\label{easyPeasy}
      %   A \times B \times C = \{ \Vector{a, \, b, \, c} \mid a \in A, \, b \in B, \, c \in C \}.
      % \end{equation}
      % В данном курсе мы будем придерживаться формального (строгого) определения, если заранее не будет оговорено, что мы имеем дело с определением вида~\eqref{easyPeasy}.

      Однако в нашем примере скобки имеются, поэтому множества:
      \[
        X \times \qty(Y \times Z) \text{ и } \qty(X \times Y) \times Z, \text{ "---}
      \]
      не имеют общих элементов, а это в свою очередь значит, что в ответе мы получим векторы из выражения~\eqref{rightCartesian}. 
  \end{enumerate}
}
\answer{%
  \begin{enumerate}[a)]%
    \item $
        \Vector{1, \, a}, \, \Vector{1, \, b}, \, \Vector{1, \, c}, \
        \Vector{2, \, a}, \, \Vector{2, \, b}, \, \Vector{2, \, c}, \
        \Vector{3, \, a}, \, \Vector{3, \, b}, \, \Vector{3, \, c}
      $.
    \item Нет элементов.
    \item $ \Vector{1, \, \Vector{\pi, \, 0}}, \Vector{2, \, \Vector{\pi, \, 0}}, \Vector{3, \, \Vector{\pi, \, 0}}, \Vector{1, \, \Vector{e, \, 0}}, \Vector{2, \, \Vector{e, \, 0}}, \Vector{3, \, \Vector{e, \, 0}} $.
  \end{enumerate}
}
%!TeX root = ./../Практика.tex
% Декартово произведение
\task{%
  Дано множество~$ \mathbb B = \{0, \, 1\} $. Для каждого натурального~$n$ найдите сумму количеств единиц в каждом из векторов множества~$\mathbb B^n$. К примеру, при~$n = 2$:
  \[
    \mathbb B^2 = \{\Vector{0, \, 0}, \, \Vector{0, \, 1}, \, \Vector{1, \, 0}, \, \Vector{1, \, 1}\},
  \]
  в сумме получаем 4 единицы.
}
\solution{%
  Заметим, что общее количество нулей и единиц для конкретного~$n$ есть ничто иное, как $ n \cdot 2^n $.
  Действительно, множество~$\mathbb B^n$ содержит $2^n$ векторов, каждый из которых состоит из $n$ элементов (нулей и единиц). К примеру, для~$ n = 3 $ имеем (8~векторов по 3~элемента в каждом):
  \[
    \begin{array}{c}%
      \underset{\color{gray} (1)}{\Vector{0, \, 0, \, 0}}, \,
      \underset{\color{gray} (2)}{\Vector{0, \, 0, \, 1}}, \,
      \underset{\color{gray} (3)}{\Vector{0, \, 1, \, 0}}, \,
      \underset{\color{gray} (4)}{\Vector{0, \, 1, \, 1}}, \\ 
      \underset{\color{gray} (5)}{\Vector{1, \, 0, \, 0}}, \,
      \underset{\color{gray} (6)}{\Vector{1, \, 0, \, 1}}, \,
      \underset{\color{gray} (7)}{\Vector{1, \, 1, \, 0}}, \,
      \underset{\color{gray} (8)}{\Vector{1, \, 1, \, 1}}. \\ 
    \end{array}
  \]

  Заметим, что в сумме для всех векторов количество нулей и единиц совпадает. И вправду, для~$n = 1$ это очевидно, а для каждого последующего~$n$ мы получаем векторы путём добавления нуля и единицы к каждому текущему вектору, то есть, к примеру, из $\mathbb B^2$ получаем следующие векторы для $\mathbb B^3$:
  \[
    \begin{array}{c}%
      \Vector{{\color{Mgreen} 0}, \, {\color{Mgreen} 0}} \to
      \Vector{{\color{Mred} 0}, \, {\color{Mgreen} 0}, \, {\color{Mgreen} 0}}, \,
      \Vector{{\color{Mred} 1}, \, {\color{Mgreen} 0}, \, {\color{Mgreen} 0}}, \\ 
      \Vector{{\color{Mgreen} 0}, \, {\color{Mgreen} 1}} \to
      \Vector{{\color{Mred} 0}, \, {\color{Mgreen} 0}, \, {\color{Mgreen} 1}}, \,
      \Vector{{\color{Mred} 1}, \, {\color{Mgreen} 0}, \, {\color{Mgreen} 1}}, \\ 
      \Vector{{\color{Mgreen} 1}, \, {\color{Mgreen} 0}} \to
      \Vector{{\color{Mred} 0}, \, {\color{Mgreen} 1}, \, {\color{Mgreen} 0}}, \,
      \Vector{{\color{Mred} 1}, \, {\color{Mgreen} 1}, \, {\color{Mgreen} 0}}, \\ 
      \Vector{{\color{Mgreen} 1}, \, {\color{Mgreen} 1}} \to
      \Vector{{\color{Mred} 0}, \, {\color{Mgreen} 1}, \, {\color{Mgreen} 1}}, \,
      \Vector{{\color{Mred} 1}, \, {\color{Mgreen} 1}, \, {\color{Mgreen} 1}}. \\ 
    \end{array}
  \]
  Значит, в результате мы получаем $ n \cdot 2^{n - 1} $ единиц.
}
\answer{%
  $ n \cdot 2^{n - 1} $.
}
%!TeX root = ./../Практика.tex
% Бинарные отношения
\task{%
  Даны множества~$A$~и~$B$:
  \[
    A = B = \{\text{{\color{Mred} камень}, {\color{Mgreen} ножницы}, {\color{Mblue} бумага}}\}.
  \]
  Определим бинарное отношение~$R$ следующим образом:
  \[
    R = \text{«побеждает»}\colon \ aRb \iff a \text{ побеждает } b.
  \]
  То есть, к примеру, $ \Vector{\text{{\color{Mred} камень}, {\color{Mgreen} ножницы}}} \in R $,
  но $ \Vector{\text{{\color{Mred} камень}, {\color{Mblue} бумага}}} \not\in R $.
  Выпишите~$R$ в явном виде (множество векторов).
  Найдите:
  \[
    \text{$\delta_R$, $\rho_R$, $R^{-1}$, $-R$, $R(\{\text{{\color{Mred} камень}, {\color{Mblue} бумага}}\})$.}
  \]
}
\solution{%
  Выпишем~$R$ в явном виде:
  \[
    R = \{
      \Vector{\text{{\color{Mred} камень}, {\color{Mgreen} ножницы}}}, \
      \Vector{\text{{\color{Mgreen} ножницы}, {\color{Mblue} бумага}}}, \
      \Vector{\text{{\color{Mblue} бумага}, {\color{Mred} камень}}}
    \}.
  \]
  
  Понятно, что~$\delta_R = \rho_R = \{\text{{\color{Mred} камень}, {\color{Mgreen} ножницы}, {\color{Mblue} бумага}}\}$,
  так как для любого элемента~$a \in A$ найдётся элемент~$b \in A$ (напомним, что~$A = B$) такой, что $ aRb $.
  
  Напомним, что обратное бинарное отношение к~$R$ есть ничто иное, как:
  \begin{equation}
    R^{-1} = \{\Vector{a, \, b} \in A \times B \mid bRa\}.
  \end{equation}
  Выпишем~$R^{-1}$:
  \[
    R^{-1} = \{
      \Vector{\text{{\color{Mgreen} ножницы}, {\color{Mred} камень}}}, \
      \Vector{\text{{\color{Mblue} бумага}, {\color{Mgreen} ножницы}}}, \
      \Vector{\text{{\color{Mred} камень}, {\color{Mblue} бумага}}}
    \}.
  \]

  Напомним, что дополнением к бинарному отношению~$R$ является:
  \begin{equation}
    -R = (A \times B) \setminus R.
  \end{equation}
  Выпишем все векторы~$-R$:
  \[
    \begin{array}{lll}%
      \Vector{\text{{\color{Mred} камень}, {\color{Mblue} бумага}}}, & 
      \Vector{\text{{\color{Mgreen} ножницы}, {\color{Mred} камень}}}, & 
      \Vector{\text{{\color{Mblue} бумага}, {\color{Mgreen} ножницы}}}, \\
      \Vector{\text{{\color{Mred} камень}, {\color{Mred} камень}}}, & 
      \Vector{\text{{\color{Mgreen} ножницы}, {\color{Mgreen} ножницы}}}, & 
      \Vector{\text{{\color{Mblue} бумага}, {\color{Mblue} бумага}}}.
    \end{array}
  \]

  Образом множества~$X$ относительно бинарного отношения~$R$ называется множество:
  \begin{equation}
    R(X) = \{b \in B \mid \exists \, a \in X : aRb\}.
  \end{equation}
  Выпишем образ множества~$ \{\text{{\color{Mred} камень}, {\color{Mblue} бумага}}\} $:
  \[
    R(\{\text{{\color{Mred} камень}, {\color{Mblue} бумага}}\}) = \{
      \Vector{\text{{\color{Mred} камень}, {\color{Mgreen} ножницы}}}, \
      \Vector{\text{{\color{Mblue} бумага}, {\color{Mred} камень}}}
    \}.
  \]
}
%!TeX root = ../Практика.tex
% Функции
\task{%
  Дано бинарное отношение~$f$:
  \begin{equation}\label{fQ}%
    f = \{
      \Vector{x, \, y} \in \R^2 \mid x \in \Q \text{ и } |x| = y
      \text{ или } x \not\in \Q \text{ и } |x| = -y
    \}.
  \end{equation}
  Определите, является ли~$f$ функцией.
  Если да, укажите~$\delta_f$,~$\rho_f$, $f^{-1}$ (при наличии);
  определите, является ли~$f$ инъекцией, сюръекцией, биекцией. 
}
\solution{%
  Напомним, что \textbf{функцией} является такое бинарное отношение~$f \in A \times B$,
  у которого~$\delta_f = A$, $\rho_f \subseteq B$, а так же
  \begin{equation}\label{uniqueY}
    \text{для всех~$x \in \delta_f$, } y_1, y_2 \in \rho_f
    \text{ из $x f y_1$ и $x f y_2$ следует } y_1 = y_2.
  \end{equation}

  Наше бинарное отношение и вправду является функцией,
  действительно, любой~$x \in \R$ либо принадлежит~$\Q$, либо не принадлежит,
  при этом что в первом, что во втором случае найдётся такой~$y \in \R$, что
  $ |x| = y \text{ или же } |x| = -y $.
  Это значит, что~$ \delta_f = A = \R $.
  Нет никаких сомнений, что $ \rho_f \subseteq B = \R $.

  Покажем, что выполняется~\eqref{uniqueY}.
  Пусть $ x f y_1 $ и $ x f y_2 $, тогда, если~$x \in \Q$:
  \[
    \qty(|x| = y_1, \ |x| = y_2) \Rightarrow y_1 = y_2.
  \]
  Если~$ x \not\in \Q $:
  \[
    \qty(|x| = -y_1 \Leftrightarrow y_1 = -|x|, \ |x| = -y_2 \Leftrightarrow y_2 = -|x|)
    \Rightarrow y_1 = y_2.
  \]
  Получается, $f$ действительно является функцией.
  График её выглядит примерно следующим образом:
  \imgl[0.45]{График функции~$f$}{Qplot.pdf}{Qplot}
  Может показаться, что область значений функции ($ \rho_f $, синяя область на графике) равна~$\R$,
  однако это не так "--- когда~$x$ является рациональным числом ($ x \in \Q $),
  $y$~принимает неотрицательные значения ($|x|$), в обратном же случае "--- отрицательные ($-|x|$).
  Иными словами, положительные значения функции рациональны, отрицательные же иррациональны.
  Получается
  \[
    \rho_f = \{y \in \Q \mid y \geq 0\} \cup \{y \in \R \setminus \Q \mid y < 0\}.
  \]
  Это в свою очередь говорит о том, что сюръекцией функция~$f$ не является.
  Более того, не является она и инъекцией, так как при значениях~$x = 1$~и~$x = -1$
  значения функции совпадают, что говорит о том, что обратной функции~($ f^{-1} $) не существует.
  А значит, не является она и биекцией.

  Может показаться очевидным, что обратной функции ($ f^{-1} $) не существует,
  это действительно так.
  Тем не менее не всё так просто "--- если в~\eqref{fQ} лишь слегка поменять условие,
  убрав модули у~$x$, мы получим уже совершенно иную функцию, график которой визуально
  совпадает с графиком на Рис.~\ref{Qplot}, но, как ни странно,
  новая функция будет являться и инъекцией, и сюръекцией, и биекцией.
  А это значит, что у неё будет и обратная функция, которая, к тому же, будет равна исходной.
}
\answer{%
  $f$ является функцией, $\delta_f = \R$,
  $ \rho_f = \{y \in \Q \mid y \geq 0\} \cup \{y \in \R \setminus \Q \mid y < 0\} $.
  Обратной функции не существует, не является ни инъекцией, ни сюръекцией, ни биекцией.
}
%!TeX root = ./../Практика.tex
% Функции
\task{%
  Для заданной функции~$ f(x)\colon \R \to \R $
  определите, является ли она инъекцией, сюръекцией, биекцией.
  \begin{enumerate}[a)]%
    \item $ f(x) = \sin [x] + \cos [x] $.
      Здесь под $ [x] $ понимается взятие целой части числа~$x$
      (то есть, к примеру,
      $ [{\color{Mgreen} 5}{\color{Mred} .999}] = {\color{Mgreen} 5}, \
        [{\color{Mgreen} -98}{\color{Mred} .1}] = {\color{Mgreen} -98} $).
    \item $ f(x) = \sgn (x) \cdot x^2 $.
      Здесь под $ \sgn (x) $ понимается функция знака
      (то есть, к примеру, $ \sgn (-7) = -1, \ \sgn (0) = 0, \ \sgn (1729) = 1 $).
  \end{enumerate}
}
\solution{%
  \begin{enumerate}[a)]%
    \item Посмотрим на график функции~$f(x)$:
      \imgt[0.75]{График функции~$ f(x) = \sin [x] + \cos [x] $}{SinPlusCosPlot.pdf}
      По графику видно, что функция ни инъекцией, ни сюръекцией не является.
      Действительно, если взять 2 различных $x$: $0$ и $0.1$, "---
      мы получим одно и тоже значение функции,
      так как целые части аргументов равны:
      \[
        [0] = [0.1] = 0 \Rightarrow
          \sin {\color{gray} \underbrace{\color{black} [0]}_{=0}}
        + \cos {\color{gray} \underbrace{\color{black} [0]}_{=0}}
        = \sin {\color{gray} \underbrace{\color{black} [0.1]}_{=0}}
        + \cos {\color{gray} \underbrace{\color{black} [0.1]}_{=0}}.
      \]
      Поэтому функция~$f(x)$ не является инъекцией.
      Сюръекцией она тоже не является, так как синус и косинус "--- ограниченные функции
      (сумма синуса и косинуса не может превышать число 2),
      то есть, область значений функции~$f(x)$ не равна~$\R$.

    \item Посмотрим на график функции~$f(x)$:
      \imgt[0.75]{График функции~$ f(x) = \sgn (x) \cdot x^2 $}{sgnXx2.pdf}
      По графику, опять же, видно, что функция~$f(x)$ является и инъекцией, и сюръекцией, а значит и биекцией.
      И правда, рассмотрим сначала интервал~$ x \in [0; \, \infty) $.
      Естественно, функция на нём инъективна:
      \[
        x \geq 0 \Rightarrow \text{ при  } x_1 \not= x_2 \quad x_1^2 \not= x_2^2.
      \]
      И, что довольно очевидно, область значений функции~$ f(x) $ на этом интервале
      есть ничто иное, как~$[0; \, \infty)$.
      При~$ x \in (-\infty; \, 0] $ ситуация совершенно идентична,
      с той лишь разницей, что область значений функции равна~$(-\infty; \, 0]$.
      
      Итого, получаем следующее: функция инъективна при $ x \in \R $,
      при этом область значений функции равна~$ \R $,
      иными словами, функция является сюръекцией, и биекцией (так как является и инъекцией, и сюръекцией).
  \end{enumerate}
}
\answer{%
  \begin{enumerate}[a)]%
    \item Не является ни инъекцией, ни сюръекцией, ни биекцией.
    \item Является и инъекцией, и сюръекцией, и биекцией.
  \end{enumerate}
}
%!TeX root = ./../Практика.tex
% Cпециальыне бинарные отношения
\task{%
  Задано бинарное отношение~$R$:
  \begin{equation}
    R = \qty{\Vector{x, \, y} \mid x, y \in \N \text{ и } x \text{ входит в запись числа } y}.
  \end{equation}
  То есть, к примеру,
  $ \Vector{{\color{Mgreen} 8}, \, {12\color{Mgreen} 8}} \in R $,
  $ \Vector{{\color{Mgreen} 10}, \, {\color{Mgreen} 10}10} \in R $,
  но $ \Vector{4, \, 123} \not\in R $,
  $ \Vector{0, \, 28} \not\in R $.
  Определите, является ли отношение~$R$: рефлексивным, иррефлексивным, симметричным, антисимметричным, транзитивным, эквивалентностью.
  % Найдите $ R^{-1} $, $ -R $, $ \delta_R $, $ \rho_R $.
}
% \footnotetext{В данной задаче не рассматриваются числа с ведущими нулями, к примеру, $ 1 = 01 = 001 = \ldots $}
\solution{%
  Бинарное отношение~$R$ называется рефлексивным, если для любого~$a \in A$ выполняется~$aRa$.
  Довольно очевидно, что любое число входит в собственную же запись, поэтому~\textit{$R$ является рефлексивным}.
  
  Иррефлексивным же отношение~$R$ называется, если при любом~$a \in A$ не выполняется~$ aRa $, что, разумеется, не так, следовательно, \textit{иррефлексивным отношение~$R$ не является}.

  Бинарное отношение~$R$ называется симметричным, если:
  \[
    \text{для всех~$a, b \in A$ } \ aRb \Rightarrow bRa.
  \]
  Понятно, что наше \textit{отношение симметричным не является}.
  Взять хотя бы, к примеру, такой вектор:
  $ \Vector{{\color{Mgreen} 13}, \, {\color{Mgreen} 13}37} \in R $, "---
  если переставить местами его элементы, получим:
  $ \Vector{1337, \, 13} \not\in R $.

  Антисимметричным является такое бинарное отношение~$R$, что:
  \[
    \text{для всех $ a, b \in A $ } \ aRb \text{ и } bRa \Rightarrow a = b.
  \]
  Наше \textit{отношение~$R$ является антисимметричным}. Действительно, пусть $ aRb $ и $ bRa $, запишем цифры числа~$a$ как~$a_1, \, \ldots, \, a_n$, цифры числа~$b$ "--- как~$b_1, \, \ldots, \, b_n$ (в~$a$~и~$b$ одинаковое количество цифр по той причине, что число с б$\acute{о}$льшим количеством цифр не может входить в число с меньшим количеством цифр), в таком случае~$a$~и~$b$ можно записать следующим образом:
  \[
    a = {\color{gray} \underbrace{\color{black} b_1\ldots b_n}_{b}}, \
    b = {\color{gray} \underbrace{\color{black} a_1\ldots a_n}_{a}}.
  \]
  Иными словами, $ a = b $.

  Бинарное отношение~$R$ называется транзитивным, если:
  \[
    \text{для всех $ a, b, c \in A $} \ aRb \text{ и } bRc \Rightarrow aRc.
  \]
  Разумеется, наше \textit{отношение является транзитивным}, так как если в записи числа~$c$ содержится~$b$, в записи которого, в свою очередь, содержится~$a$, то~$a$ содержится и в записи~$c$. Для примера возьмём три числа:
  \[
    a = {\color{Mgreen} 2,} \
    b = {\color{Mred} 1}{\color{Mgreen} 2}{\color{Mred} 4}, \
    c = 0{\color{Mred} 1}{\color{Mgreen} 2}{\color{Mred} 4}8.
  \]

  Рефлексивное, симметричное, транзитивное бинарное отношение на~$A$ называется эквивалентностью на~$A$. Наше отношение не является симметричным, а значит и \textit{эквивалентностью оно не является}.
}
\answer{%
  $R$ является рефлексивным, антисимметричным, транзитивным. Не является иррефлексивным, симметричным, эквивалентностью.
}
%!TeX root = ./../Практика.tex
% Специальные бинарные отношения
\task{%
  Определите, является ли эквивалентностью бинарное отношение~$R$. Если да, то опишите соответствующее фактор-множество.
  \begin{enumerate}[a)]%
    \item $ R = \{ \Vector{a, \, b} \in \N^2 \mid a = b \text{ или } a^2 + b^2 \text{ "--- простое число} \} $.
    \item $ R = \{ \Vector{a, \, b} \in \N^2 \mid a \cdot b \text{ "--- квадрат некоторого числа } n \in \N \} $.
  \end{enumerate}
}
\solution{%
  \begin{enumerate}[a)]%
      \item Довольно очевидно, что наше отношение является рефлексивным и симметричным
        (это понятно из условия «$ a = b $» и ассоциативности сложения).
        Вопрос заключается в том, является ли оно транзитивным.
        Оказывается, не является "--- взять хотя бы такой пример:
        \[
          \begin{array}{rcll}%
            {\color{Mred} 1}^2 + {\color{Mgreen} 2}^2 & = & 5 &  \text{ "--- простое число}, \\ 
            {\color{Mgreen} 2}^2 + {\color{Mblue} 3}^2 & = & 13 & \text{ "--- простое число}, \\ 
            {\color{Mred} 1}^2 + {\color{Mblue} 3}^2 & = & 10 & \text{ "--- составное число}. \\ 
          \end{array}
        \]
        А это значит, что и эквивалентностью~$R$ не является.

      \item Рефлексивность данного отношения, опять же, очевидна
        ($a^2$ "--- всегда квадрат числа~$a$, то есть, всегда выполняется~$aRa$).
        То же касается и симметричности (объясняется это ассоциативностью умножения).

        Проверим, является ли данное отношение транзитивным.
        Запишем числа~$a$,~$b$~и~$c$
        в виде бесконечного произведения простых чисел в некоторых целых степенях:
        \begin{equation}\label{primes}
          \begin{array}{rcl}%
                   {\color{Mred} a}
             & = & 2^{{\color{Mred} x_1}} \cdot
                   3^{{\color{Mred} x_2}} \cdot
                   5^{{\color{Mred} x_3}} \cdot
                   7^{{\color{Mred} x_4}} \cdot \ldots, \\ 
                   {\color{Mgreen} b}
             & = & 2^{{\color{Mgreen} y_1}} \cdot
                   3^{{\color{Mgreen} y_2}} \cdot
                   5^{{\color{Mgreen} y_3}} \cdot
                   7^{{\color{Mgreen} y_4}} \cdot \ldots, \\ 
                   {\color{Mblue} c}
             & = & 2^{{\color{Mblue} z_1}} \cdot
                   3^{{\color{Mblue} z_2}} \cdot
                   5^{{\color{Mblue} z_3}} \cdot
                   7^{{\color{Mblue} z_4}} \cdot \ldots \\   
          \end{array}
        \end{equation}
        Покажем, к примеру, как записать подобным образом числа 1 и 36
        ($ 36 = 2^{{\color{Mred} 2}} \cdot 3^{{\color{Mred} 2}} $):
        \[
          1  = 2^0 \cdot 3^0 \cdot 5^0 \cdot 7^0 \cdot \ldots, \ 
          36 = 2^{{\color{Mred} 2}} \cdot 3^{{\color{Mred} 2}} \cdot 5^0 \cdot 7^0 \cdot \ldots
        \]
        Заметим, что квадратом любого натурального числа~$n$ будет такое число,
        все степени в записи~\eqref{primes} которого будут чётными
        (так как возведение в квадрат числа~$n$ равносильно умножению на 2 степеней простых чисел).

        Вспомним, что при умножении степени складываются.
        Предположим, что~$aRb$~и~$bRc$, тогда получим:
        \[
          \begin{array}{rcll}%
            ({\color{Mred} x_1} + {\color{Mgreen} y_1}) & \text{ и } &
            ({\color{Mgreen} y_1} + {\color{Mblue} z_1}) & \text{ "--- чётные числа}, \\ 
            ({\color{Mred} x_2} + {\color{Mgreen} y_2}) & \text{ и } &
            ({\color{Mgreen} y_2} + {\color{Mblue} z_2}) & \text{ "--- чётные числа}, \\ 
            ({\color{Mred} x_3} + {\color{Mgreen} y_3}) & \text{ и } &
            ({\color{Mgreen} y_3} + {\color{Mblue} z_3}) & \text{ "--- чётные числа}, \\ 
            ({\color{Mred} x_4} + {\color{Mgreen} y_4}) & \text{ и } &
            ({\color{Mgreen} y_4} + {\color{Mblue} z_4}) & \text{ "--- чётные числа}, \\ 
            \vdots & \vdots & \vdots & \\ 
          \end{array}
        \]

        А теперь вспомним арифметику:
        если при сложении двух чисел~$p$~и~$q$ мы получили чётное число~$w$,
        то числа $p$~и~$q$ обязаны быть либо оба чётными, либо оба нечётными.
        Получается, числа~${\color{Mred} x_k}$,~${\color{Mgreen} y_k}$~и~${\color{Mblue} z_k}$
        обязаны иметь одинаковую чётность при любом натуральном~$k$,
        а это в то же время значит, что $ {\color{Mred} x_k} + {\color{Mblue} z_k} $
        всегда будет чётным числом.
        Выходит, что наше отношение является транзитивным, а значит, является и эквивалентностью.

        В таком случае, нам осталось лишь выписать соответствующее фактор-множество.
        Для начала напомним, что классом эквивалентности~$a \in A$ по~$R$ называется множество~$ a / R $:
        \begin{equation}
          a / R = \{b \in B \mid bRa\}.
        \end{equation}
        Фактор-множеством (обозначим его как~$F$) же является множество всех классов эквивалентности.
        Понятно, что для любого числа~$a$, если записать его как в выражении~\eqref{primes},
        $ a / R $ будет состоять из чисел,
        степени простых чисел у которых будут совпадать по кратности
        со степенями соответствующих простых чисел из~$a$.
        Выходит, нам нужно просто перебрать все возможные кратности степеней.
        Таким образом мы получим следующие классы эквивалентности:
        \[
          F_{{\color{Mgreen} i_1}, {\color{Mgreen} i_2}, {\color{Mgreen} i_3}, \ldots} = \qty{
            2^{{\color{Mgreen} i_1} + 2{\color{Mred} k_1}} \cdot
            3^{{\color{Mgreen} i_2} + 2{\color{Mred} k_2}} \cdot
            5^{{\color{Mgreen} i_3} + 2{\color{Mred} k_3}} \cdot \ldots
            \mid
            {\color{Mred} k_1},
            {\color{Mred} k_2},
            {\color{Mred} k_3}, \ldots \in \N_0
          },
        \]
        где каждое из~$ {\color{Mgreen} i_n} $ будет принимать значения 0 и 1
        (что меняет кратность степени).
        Тогда фактор-множество~$F$ будет равно:
        \[
          F = \qty{
            F_{{\color{Mgreen} i_1}, {\color{Mgreen} i_2}, {\color{Mgreen} i_3}, \ldots}
            \mid
            {\color{Mgreen} i_1}, {\color{Mgreen} i_2}, {\color{Mgreen} i_3}, \ldots \in \{0, \, 1\}
          }.
        \]
    \end{enumerate}
}
\answer{%
  \begin{enumerate}[a)]%
    \item Не является эквивалентностью.
    \item Является эквивалентностью. Фактор-множество (F):
      \[
        F = \qty{
          F_{i_1, i_2, i_3, \ldots}
          \mid
          i_1, i_2, i_3, \ldots \in \{0, \, 1\}
        }, \text{ где}
      \]
      \[
        F_{i_1, i_2, i_3, \ldots} = \qty{
          2^{i_1 + 2k_1} \cdot
          3^{i_2 + 2k_2} \cdot
          5^{i_3 + 2k_3} \cdot \ldots
          \mid
          k_1,
          k_2,
          k_3, \ldots \in \N_0
        }.
      \]
  \end{enumerate}
}


% Отношения поярдка
\task{%
  Дано бинарное отношение~$R$.
  Укажите, является ли оно предпорядком,
  частичным порядком,
  линейным порядком,
  полным линейным порядком.
  Если является линейным порядком, укажите минимальный и максимальный элементы (при наличии).
  \begin{enumerate}[a)]%
    \item $ R = \{ \Vector{a, \, b} \in \R^2 \mid \cos^2 a + \sin^2 b = 1 \} $.
    \item $ R = \{ \Vector{a, \, b} \in \R^2 \mid \sgn (a - b) + 1 \not= 0 \} $.
  \end{enumerate}
}
\solution{%
  Напомним следующие определения:
  \begin{itemize}%
    \item Бинарное отношение~$R$ называется \textbf{предпорядком}, если оно рефлексивно и транзитивно.
    \item Предпорядок называется \textbf{частичным порядком}, если он антисимметричен.
    \item Частичный порядок называется \textbf{линейным},
      если для любых элементов~$a, b \in A$ либо~$ a \leq b $,
      либо~$ b \leq a $.
    \item Линейный порядок на множестве~$A$ называется \textbf{полным},
      если каждое непустое подмножество множества~$A$ имеет наименьший элемент.  
  \end{itemize}
  \begin{enumerate}[a)]%
    \item Ясно, что наше отношение является рефлексивным
      (ввиду основного тригонометрического тождества~$ \cos^2 x + \sin^2 x = 1 $),
      Покажем, что оно является и транзитивным.
      Пусть~$aRb$~и~$bRc$, тогда:
      \[
        \begin{array}{rl}%
            & \cos^2 {\color{Mred} a} + \sin^2 {\color{Mgreen} b} = 1, \\
            & \cos^2 {\color{Mgreen} b} + \sin^2 {\color{Mblue} c} = 1, \\
            & \cos^2 {\color{Mred} a} + \sin^2 {\color{Mblue} c} = \\ 
          = & (1 - \sin^2 b) + (1 - \cos^2 b) = \\ 
          = & 2 - (\cos^2 b + \sin^2 b) = \\ 
          = & 1. \\ 
        \end{array}
      \]
      Значит, из~$aRb$~и~$bRc$ следует~$aRc$.
      Получается, что наше отношение является предпорядком.
      Тем не менее, антисимметричным~$R$ не является.
      Действительно, если для любого~$a$ взять~$b = a + 2\pi$,
      то мы получим~$aRb$~и~$bRa$, но в то же время~$a \not= b$.

      Итого, наше отношение не является ни частичным порядком,
      ни линейным порядком,
      ни полным линейным порядком.

    \item .
  \end{enumerate}
}


\clearpage

\subtitle{Контрольные вопросы}

%!TeX root = ./Практика.tex
\begin{enumerate}%
  \item При каких множествах~$A$,~$B$~(и~$C$) верно равенство?
    \begin{enumerate}[a)]%
      \item $ A \times B = B \times A $.
      \item $ (A \times B) \times C = A \times (B \times C) $.
    \end{enumerate}


  \item Может ли бинарное отношение совпадать со своим дополнением? С обратным отношением?
  
  \item Может ли функция $ f(x)\colon \R \to \R $ быть биекцией,
    если на некоторых интервалах она убывающая, а на других "--- возрастающая?

  \item Может ли функция~$ f\colon \N \to \N $ являться сюръекцией,
    не являясь при этом инъекцией? 

  \item Если $A$ "--- множество из $n$~элементов ($ n \in \N $), 
    а~$U$ "--- множество всех функций из~$A$ в~$A$,
    то каких функций в~$U$ больше: инъективных или сюръективных?
    Или же, может быть, их одинаковое количество? 
  
  \item Может ли бинарное отношение одновременно являться симметричным и антисимметричным?
    Рефлексивным и иррефлексивным?
\end{enumerate}


\subtitle{Литература}

\begin{enumerate}%
  \item Пак~В.\,Г. «Сборник задач по дискретной математике. Теория Множеств. Комбинаторика», Балт. гос. техн. ун-т. "--- СПб., 2008. "--- 118\,с.
  \item Куратовский~К., Мостовский~А. «Теория множеств», издательство «Мир» "--- М., 1970. "--- 416\,с.
\end{enumerate}